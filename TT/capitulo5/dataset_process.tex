\newpage
\section{Conjunto de entrenamiento: CROHME}

El conjunto de datos de entrenamiento para el desarrollo de la red es el denominado \textbf{CROHME} por sus siglas en inglés: \textbf{Competition on Recognition of Online Handwritten Mathematical Expressions} publicado por los organizadores de la competencia internacional CROHME; para el caso del conjunto de entrenamiento fue posible reunir 7169 elementos que de acuerdo a investigación previa \cite{chino}, son relativamente pocos  para esperar una buena precisión, los  elementos en en el conjunto son archivos de tipo INKML.
\subsection{Formato del conjunto de datos}

Como ya se mencionó, los elementos del conjunto son archivos de tipo INKML (Ink Markup Language) y que principalmente se compone de tres partes:

\begin{itemize}
	\item Ink: Un conjunto de trazos definidos por puntos.
	\item Nivel de símbolo Ground-Truth: La segmentación e información de etiqueta de cada símbolo de la expresión.
	\item Ground-truth: La estructura MATHML de la expresión.
\end{itemize}

La información de Ground-Truth tanto de nivel de símbolo como de la expresión matemática fueron ingresadas manualmente por los colaboradores de la generación del conjunto, además, alguna información general es agregada en el archivo:

\begin{itemize}
	\item Los canales (en este caso X, Y)
	\item La información del escritor (Identificación, entrega, edad, Mano dominante, género, etc ), si está disponible.
	\item Ground-Truth en \LaTeX{} para fácilmente renderizarlo.
\end{itemize}

A continuación se muestra un ejemplo de un archivo del dataset que representa la expresión \$2 \^\ \{-1 \} \$:\\
 $2^{-1}$ renderizado.
\lstinputlisting[language=XML]{capitulo5/dataset/Inkdata_temp_InkFR_HPR_EQU_NOC_scc696_fi6_db144195.inkml}
Sin embargo, para el propósito del trabajo terminal, el conjunto de datos no es útil en este formato, por lo que se tenía que transformar la información de los trazos en imágenes.
\subsection{Conversión a imagen}
Con la información de los trazos es posible generar una imagen con los puntos de los trazos en negro y fondo blanco, el primer reto fue identificar las etiquetas que contenían a los trazos, para ello se utilizó la biblioteca de Python xml.etree y una función de terceros para poder utilizar dichos trazos posteriormente:

\lstinputlisting[language=Python]{capitulo5/dataset/getTraces.py}

Una vez obtenidos los trazos y con ayuda de matplotlib fueron separados como puntos $x,y$ y utilizados en la función plot de matplotlib para posteriormente guardarlo como imagen.

\lstinputlisting[language=Python]{capitulo5/dataset/inkml2img_short.py}

Esto tenía que realizarse por cada uno de los elementos del conjunto de entrenamiento, además de también extraer la expresión matemática encerrada entre las etiquetas \textbf{$<$annotation$><$/annotation$>$} con el atributo \textbf{type}, para ello nuevamente se utilizó la biblioteca de python xml.etree para acceder a los nodos del árbol directamente:
\lstinputlisting[language=Python]{capitulo5/dataset/tag.py}
Con estos subscripts fue posible desarrollar una biblioteca que permite acceder a la carpeta con el conjunto de entrenamiento en formato inkml y guardarlos como imagen en otra carpeta junto con un archivo CSV conteniendo la ruta relativa de la imagen y la expresión en \LaTeX{} correspondiente separados por coma.

\lstinputlisting[language=Python, firstline = 1, lastline=10]{capitulo5/dataset/training.csv}
\subsection{Generador de secuencia de tokens}
El archivo CSV generado con lo descrito anteriormente no es suficiente para cargarlo en TensorFlow, la expresión matemática en \LaTeX{} debía ser expresada como una secuencia numérica, por lo que se necesitaba identificar a cada símbolo con un número único y conformar a la secuencia, de modo que la estructura del archivo CSV pasaría de tener la forma \textbf{RUTA\_IMAGEN}, \textbf{EXPRESION\_LATEX} a tener la forma \textbf{RUTA\_IMAGEN}, \textbf{SECUENCIA\_NUMÉRICA}, teniendo así una nueva representación del conjunto de entrenamiento conformada por una carpeta con las imágenes y un archivo CSV previamente descrito para poder cargarse en TensorFlow.
\newpage
Para esto se desarrolló gracias a lex en Python un script para especificar los tokens tomando como base los símbolos especificados en la sección \ref{sec:del_exp}.\\\\
\lstinputlisting[language=Python, lastline=10]{capitulo5/dataset/Rules.py}

\lstinputlisting[language=Python, firstline = 26, lastline=44]{capitulo5/dataset/Analyzer.py}

\lstinputlisting[firstline = 1, lastline=10]{capitulo5/dataset/tokenized_training.csv}
\newpage
Es importante destacar que se debe también guardar este mapeo único y no alterar el orden, por lo que de requerir agregar nuevos símbolos al conjunto es necesario hacerlo al final de los ya existentes, ya que la red entrenada devuelve secuencias numéricas que deben mapearse para tener la correspondiente expresión en \LaTeX{}.
%\subsubsection{}