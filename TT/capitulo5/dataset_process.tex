\newpage
\section{Conjunto de entrenamiento: CROHME}

El conjunto de datos de entrenamiento para el desarrollo de la red es el denominado \textbf{CROHME} por sus siglas en inglés: \textbf{Competition on Recognition of Online Handwritten Mathematical Expressions} publicado por los organizadores de la competencia internacional CROHME; para el caso del conjunto de entrenamiento fue posible reunir 7169 que de acuerdo a investigación previa \cite{chino}, son muy pocos elementos  para esperar una buena precisión, los  elementos en en el conjunto son archivos de tipo INKML.
\subsection{Formato del conjunto de datos}

Como ya se mencionó, los elementos del conjunto son archivos de tipo INKML (Ink Markup Language) y que principalmente se compone de tres partes:

\begin{itemize}
	\item Ink: Un conjunto de trazos definidos por puntos.
	\item Nivel de símbolo Ground-Truth: La segmentación e información de etiqueta de cada símbolo de la expresión.
	\item Ground-truth: La estructura MATHML de la expresión.
\end{itemize}

La información de Ground-Truth tanto de nivel de símbolo como de la expresión matemática fueron ingresadas manualmente por los colaboradores de la generación del conjunto, además, alguna información general es agregada en el archivo:

\begin{itemize}
	\item Los canales (en este caso X, Y)
	\item La información del escritor (Identificación, entrega, edad, Mano dominante, género, etc ), si está disponible.
	\item Ground-Truth en \LaTeX{} para fácilmente renderizarlo.
\end{itemize}

A continuación se muestra un ejemplo de un archivo del dataset que representa la expresión \$2 \^\ \{-1 \} \$:\\
 $2^{-1}$ renderizado.
\lstinputlisting[language=XML]{capitulo5/dataset/Inkdata_temp_InkFR_HPR_EQU_NOC_scc696_fi6_db144195.inkml}
Sin embargo, para el propósito del trabajo terminal, el conjunto de datos no es útil en este formato, por lo que se tenía que transformar la información de los trazos en imágenes.
\subsection{Conversión a imagen}
Con la información de los trazos es posible generar una imagen con los puntos de los trazos en negro y fondo, el primer reto fue identificar las etiquetas que contenían a los trazos, para ello se utilizó la biblioteca de Python xml.etree y una función de terceros para poder utilizar dichos trazos posteriormente:

\lstinputlisting[language=Python]{capitulo5/dataset/getTraces.py}

Una vez obtenidos los trazos y con ayuda de matplotlib fueron separados como puntos $x,y$ y utilizados en la función plot de matplotlib para posteriormente guardarlo como imagen.

\lstinputlisting[language=Python]{capitulo5/dataset/inkml2img_short.py}

Esto tenía que realizarse por cada uno de los elementos del conjunto de entrenamiento, además de también extraer la expresión matemática encerrada entre las etiquetas \textbf{$<$annotation$><$/annotation$>$} con el atributo \textbf{type}, para ello nuevamente se utilizó la biblioteca de python xml.etree para acceder a los nodos del árbol directamente:
\lstinputlisting[language=Python]{capitulo5/dataset/tag.py}
Con estos subscripts fue posible desarrollar una biblioteca que permite acceder a la carpeta con el conjunto de entrenamiento en formato inkml y guardarlos como imagen en otra carpeta junto con un archivo CSV conteniendo la ruta relativa de la imagen y la expresión en \LaTeX{} correspondiente separados por coma.
\subsection{Secuencia de tokens}

%\subsubsection{}
