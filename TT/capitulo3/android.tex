\subsection{Android}
Android es un sistema operativo para móviles que en la actualidad es desarrollado por Google. Esta principalmente pensado para dispositivos con pantalla táctil como smartphones y tablets. Y que en la actualidad junto con iOS son las principales opciones en cuanto a sistemas operativos para teléfonos móviles. Es por esto que en la Tabla \ref{tbl:comparativa-moviles} se hace una comparativa de las principales características que se tomaron en cuenta para la elección de sistema operativo de dispositivos móviles que se usaría en este trabajo.
\begin{center}
    \begin{longtable}{|J{3cm}|J{5cm}|J{5cm}|}
    \hline
    \textbf{Características} & \textbf{Android} & \textbf{iOS} \\ \hline
    Núcleo & UNIX & UNIX \\ \hline
    Lenguajes de desarrollo & Java, C, C++, Kotlin & Swift, C, C++, Objective-C \\ \hline
    Ambiente de desarrollo & Android Studio, disponible en Windows, Linux y MacOS & Xcode, solo disponible en MacOS \\ \hline
    Complejidad de desarrollo & Medio complejo debido a la gran cantidad de diferencias entre dispositivos y versiones de sistema operativo & Poco compleja debido a que hay poca diversidad de versiones sistemas operativos y dispositivos \\ \hline
    Tiempo de desarrollo & Suele tomar 30\%-40\% más que para iOS \cite{ddi} & Depende de la complejidad del desarrollo \\ \hline
    Tiempo de despliegue & Es rápido desplegar una aplicación de Android debido a los test automáticos que se realizan sobre esta para su verificación & Es lento su despliegue debido a que la verificación es manual \\ \hline
    Cantidad del mercado & La cantidad de usuarios es mayor, tan solo el primer cuarto del 2017 el 86.1\% de los teléfonos vendidos fueron Android \cite{ddi} & Su mercado es menor al de Android \\ \hline
    Código abierto & El kernel, UI y algunas aplicaciones estándar son de código abierto & El kernel no es de código abierto pero esta basado en Darwin OS que es de código abierto \\ \hline
    Ultima versión & Android 10 (Septiembre 3, 2019) & iOS 13 (Septiembre 19, 2019) \\ \hline
    Seguridad & Actualizaciones de seguridad bastante regulares. Sin embargo, debido a los fabricantes dichas actualizaciones pueden demorarse en ser aplicadas. También se pueden instalar aplicaciones externas a la Play Store por lo que esto puede generar problemas de seguridad & Pocas actualizaciones, las amenazas de seguridad son pocas debido a que descargar aplicaciones fuera de la App Store es complicado. \\ \hline
    \caption{Tabla comparativa de sistemas operativos de dispositivos móviles}
    \label{tbl:comparativa-moviles}
    \end{longtable}
\end{center}

Android nos brinda un ambiente de desarrollo más flexible que el que presenta iOS debido a las características listadas con anterioridad. Es cierto que el desarrollo puede ser más tardado sin embargo para este proyecto el optar por iOS nos generaría problemas debido a que no solo la plataforma de desarrollo es más cerrada sino que el tiempo de desarrollo seria mayor debido a la curva de aprendizaje de Swift y Objective-C a la cual nos enfrentaríamos.

Finalmente, la cantidad del mercado que tiene Android es significativamente mayor que la que tiene iOS por ende este es un gran punto a considerar. Es por todas estas cuestiones que se opto por desarrollar el trabajo para la plataforma de Android.