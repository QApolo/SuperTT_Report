\section{Requerimientos funcionales}
    Los requerimientos funcionales describen los actividades y comportamientos que tendrá el sistema bajo ciertas condiciones, de igual forma pueden declarar lo que el sistema no debe de hacer.
    
    En esta sección se presentan los requerimientos funcionales que se obtuvieron para el sistema. Dichos requerimientos se encuentran separados de acuerdo a los diferentes módulos que se tienen planeados.
    \subsection{Módulo de usuarios}
    \begin{enumerate}[label=\textbf{RF\arabic*.}]
    \item \textbf{Mecanismo de gestión de usuarios.} Proporcionar al usuario la posibilidad de creación, consulta y modificación de los datos de su cuenta de usuario. Estas operaciones debe de estar presentes en la aplicación web y de android y solo serán permitidas para usuarios con cuenta verificada.
    \item \textbf{Mecanismo de autenticación de usuarios.} Proporcionar un mecanismo para el inicio y cierre de sesión de la cuenta del usuario cuya cuenta haya sido verificada en la aplicación web y android.
    \item \textbf{Mecanismo de verificación de cuenta.} Proporcionar al usuario una forma para verificar que una cuenta creada es valida a través de una verificación basada en el envió de un correo electrónico para la confirmar dicha validez y hacer uso del resto de funcionalidad del sistema. Esta verificación solo se podra realizar a través de la aplicación web.
    \item \textbf{Mecanismo de recuperación de contraseñas.} Proporcionar al usuario la posibilidad de recuperar la contraseña asociada a su cuenta a través de un envío de correo electrónico con el cual podrá acceder a una interfaz en la aplicación web para recuperar el acceso a su cuenta. La recuperación de contraseñas estará disponible en la aplicación android y web.
    \item \textbf{Mecanismo para la comunicación entre la aplicación android y web.} El sistema debe de contar con una interfaz para la comunicación entre las dos aplicaciones que se tienen, la forma en que se realizara sera un API REST y el formato para el envió de información sera JSON.
    \item \textbf{Mecanismo para la autenticación en el API REST.} El sistema debe de brindar una forma de garantizar que la comunicación entre la aplicación web y android es seguro mediante el uso de tokens de autenticación en cada petición y respuesta que se realice.
    \end{enumerate}
    
    \subsection{Módulo de proyectos}
    \begin{enumerate}[label=\textbf{RF\arabic*.}]
    	\setcounter{enumi}{4}
    	\item \textbf{Mecanismo para la gestión de proyectos de \LaTeX{}.} Proporcionar al usuario un mecanismo para visualizar, editar, crear o borrar proyectos asociados a su cuenta. La gestión se podrá realizar en la aplicación web y android.
    	\item \textbf{Permitir descargar el archivo de \LaTeX{}} que se haya traducido. Proporcionar al usuario la funcionalidad de generar un archivo \LaTeX{} que contenga la traducción de expresiones matemáticas que se encuentren en un proyecto. La descarga de dicho archivo solo estará disponible en la aplicación web.
    	\item \textbf{Permitir calificar una traducción realizada.} Proporcionar al usuario la funcionalidad de calificar que tan buena fue la traducción de una expresión matemática en una imagen para brindar retroalimentación, dicha calificación podrán ser valores enteros entre 1 y 5. Para poder calificar una traducción se tendrá que hacer uso de la aplicación web.
    	\item \textbf{Permitir el uso de la cámara del dispositivo android para tomar fotografías.} Proporcionar un mecanismo en la aplicación web que permita acceder a la cámara del dispositivo, tomar una fotografía, visualizarla.
    	\item \textbf{Permitir la visualización del resultado de la traducción.} Proporcionar al usuario una interfaz en la cual pueda observar la traducción a \LaTeX{} que se realizó a partir de una imagen. Esto se podrá hacer en la aplicación web y android.
    	\item \textbf{Permitir añadir al portapapeles alguna traducción seleccionada.} Proporcionar al usuario la funcionalidad para copiar el código de una traducción a su portapapeles para su posterior uso. Esta funcionalidad está limitada a solo la aplicación web.
    	\item \textbf{Mecanismo para el envío de imágenes tomadas por la aplicación android a la aplicación web para su uso.} El sistema debe de tener un mecanismo que a través del uso del API REST permita a la aplicación android el enviar una imagen a la aplicación web para que esta sea tratada. 
    \end{enumerate}
    
    \subsection{Módulo de análisis}
    \begin{enumerate}[label=\textbf{RF\arabic*.}]
    	\setcounter{enumi}{11}
    	\item \textbf{Mecanismo para el tratamiento de la imagen previo a su análisis.} El sistema deberá de realizar un tratamiento al imagen que reciba de la aplicación android para hacer que esta sea más sencilla de trabajar destacando sus partes importantes.
    	\item \textbf{Mecanismo para el reconocimiento de un conjunto definido de expresiones matemáticas en imágenes.} El sistema deberá de reconocer ciertas expresiones en particular con la posibilidad de que expresiones que no se encuentren en este conjunto produzcan resultados no esperados.
    \end{enumerate}
    
    \subsection{Módulo de traducción}
    \begin{enumerate}[label=\textbf{RF\arabic*.}]
    	\setcounter{enumi}{13}
    	\item \textbf{Mecanismo para la traducción a \LaTeX{} de las expresiones matemáticas encontradas en el modulo de análisis.} Implementar un algoritmo basado de gramáticas, redes neuronales o arboles de recubrimiento mínimo que permita transformar la salida que proporcione el módulo de análisis a código \LaTeX que pueda ser interpretado por un compilador.
    \end{enumerate}
    
\section{Requerimientos No Funcionales} 

 %   A continuación se enlistan los requerimientos no funcionales.
 
 \begin{center}
 	\begin{tabularx}{1.0\textwidth} { 
 			| >{\raggedright\arraybackslash}X 
 			| >{\arraybackslash}X 
 			| >{\raggedright\arraybackslash}X | }
 		\hline
 		
 		Requerimiento No Funcional & Descripción & Métricas asociadas  \\
 		\hline
 		RNF1 Seguridad  &  La seguridad se incrementa al realizar un cifrado de las contraseñas de los usuarios así como una verificación de la cuenta del usuario para poder hacer uso del sistema. Por otro lado, en la comunicación que se realiza entre la aplicación móvil y el API REST se utiliza un token para autenticar las peticiones que se realizan, dicho token es único para cada usuario.  &    Al utilizar el framework de desarrollo Django se cubren los aspectos de seguridad como vulnerabilidades tales como SQL injection, CSRF, CSS, Clickjacking y Session Hijacking, ataques JWT son manejados por autenticación por acceso limitado con el uso de oauth2 que define 4 tipos de autorización: 
 		\begin{itemize}
 			\item Código de autorización: usado con aplicaciones del lado del servidor
 			\item Implícito:utilizado con aplicaciones móviles o aplicaciones web (aplicaciones que se ejecutan en el dispositivo del usuario).
 			
 			\item Credenciales de contraseña del propietario del recurso: utilizado con aplicaciones confiables, como aquellas pertenecientes al servicio
 			\item Credenciales del cliente: Usadas con el acceso API de aplicaciones.
 		\end{itemize} \\
 		\hline
 		
 	\end{tabularx}
 	\begin{tabularx}{1.0\textwidth} { 
 			| >{\raggedright\arraybackslash}X 
 			| >{\arraybackslash}X 
 			| >{\raggedright\arraybackslash}X | }
 		
 		\hline
 		RNF2 Usabilidad  & La interfaz de usuario debe de ser intuitiva con el objetivo de hacer uso de las funcionalidades del sistema de una forma fácil para el usuario.  & La aplicación android y web fueron desarrollados utilizando material design que respalda con métricas de Actitud y Comportamiento de User Experience (UX) tales como:
 		\begin{itemize}
 			\item Abandonment Rate
 			\item Usability
 			\item Appearance
 			\item credibility
 		\end{itemize}  \\
 		\hline
 		RNF3 Escalabilidad  &   El sistema deberá ser fácilmente escalable y con ello poseer la cualidad de que si se agregan nuevas funcionalidades, estas sean fáciles de acoplar con lo ya desarrollado. Además, de que debe de ser posible aumentar sus capacidades para brindar servicio a más usuarios.
 		& La escalabilidad puede llevarse a cabo mediante los proveedores de servicios de Cloud que implementan el denominado Auto Scaling que es un sub-sistema propio de cada servicio de Cloud con la función de ajustar automáticamente la capacidad para mantener un desempeño predecible y estable, sin mencionar que también se dispone de medidores de recursos gráfico.  \\
 		\hline
 		RNF4 Disponibilidad  &   El sistema debería estar disponible en la mayor parte del tiempo, para lograr esto el uso de un proveedor de hosting es necesario. Entre las posibles opciones están Amazon Web Services, Azure o Google Cloud
 		&   Actualmente los proveedores de servicios de Cloud ofrecen una disponibilidad que se mide como "Monthly Uptime Percentage" que se calcula al sustraer del 100\% el porcentaje de minutos durante el mes se encuentra en estado "Region No Disponible" garantizando un porcentaje mayor al 99\% mensual pero menor al 99.9\%. \\
 		\hline
 	\end{tabularx}
 \end{center}
 
            
\section{Requerimientos técnicos}
    A continuación se en listan los requerimientos técnicos para un mejor funcionamiento del sistema:
    \subsection{Aplicación móvil}
    \subsubsection{Requerimientos mínimos de software}
    \begin{enumerate}
        \item Sistema operativo Android 4.5 o superior.
        \item Conexión a internet.
    \end{enumerate}
    
    \subsubsection{Requerimientos mínimos de hardware}
    \begin{enumerate}
        %\item Resolución cámara: 13 Megapixeles
        %\item Procesador: Dualcore de 1.2 GHz.
        \item Memoria RAM: 2 GB.
        \item Espacio de almacenamiento de 50 MB.
        \item Cámara de al menos 8 Megapixeles.
    \end{enumerate}
    \subsection{Aplicación web}
    \subsubsection{Requerimientos mínimos de software}
    \begin{enumerate}
        \item Cualquiera de las versiones de los siguientes navegadores hasta su versión más reciente.
        \begin{itemize}
            \item Google Chrome 7
            \item Edge 18
            \item Internet Explorer 11
            \item Firefox 4
            \item Safari 5
            \item Opera 12.1
            \item iOS Safari 13.1
            \item Chrome for android 76
            \item Firefox para android 68
        \end{itemize}
    \end{enumerate}
    \subsubsection{Requerimientos mínimos de hardware}
    \begin{enumerate}
        \item Memoria RAM: 2 GB.
        \item Espacio de almacenamiento de 100 MB.
    \end{enumerate}
    %Poner los recomendados?