\chapter{Análisis del sistema}

%Todo el análisis previo al diseño y desarrollo del sistema, hasta cierto punto lo más de hueva que se tienen que hacer pero que fundamenta el porque realizar el trabajo. Las secciones que deberiamos cubrir en este apartado son las siguientes:
%
%\begin{itemize}
%    \item Factibilidad
%        \begin{itemize}
%            \item técnica
%            \item operativa
%            \item económica
%            \item analisis de factibilidad (conclusiones con base en los tres puntos anteriores)
%        \end{itemize}
%    \item Requerimientos
%        \begin{itemize}
%            \item funcionales
%            \item no funcionales
%            \item operativos o de sistema
%        \end{itemize}
%    \item Reglas de negocio
%    \item Análisis de riesgos
%    \item Descripción del software
%    \item Herramientas de desarrollo
%    \item Metodología
%\end{itemize}

\section{Requerimientos funcionales}
    Los requerimientos funcionales describen los actividades y comportamientos que tendrá el sistema bajo ciertas condiciones.
    
    En esta sección se presentan los requerimientos funcionales que se obtuvieron para el sistema. Dichos requerimientos se encuentran separados de acuerdo a los diferentes módulos que se tienen planeados
    \subsection{Módulo de usuarios}
    \begin{enumerate}[label=RF\arabic*.]
    \item Mecanismo de gestión de usuarios.
    \item Mecanismo de autenticación de usuarios
    \item Mecanismo de recuperación de contraseñas.
    \item Mecanismo para la comunicación entre la aplicación android y la aplicación web.
    \end{enumerate}
    
    \subsection{Módulo de proyectos}
    \begin{enumerate}[label=RF\arabic*.]
    	\setcounter{enumi}{4}
    	\item Mecanismo para la gestión de proyectos de \LaTeX{}.
    	\item Permitir descargar el archivo de \LaTeX{} que se haya traducido.
    	\item Permitir calificar una traducción realizada.
    	\item Permitir el uso de la capara del dispositivo android para tomar fotografías.
    	\item Permitir la visualización del resultado de la traducción.
    	\item Permitir añadir al portapapeles alguna traducción seleccionada.
    	\item Mecanismo para el envío de imágenes tomadas por la aplicación android a la aplicación web para su uso.
    \end{enumerate}
    
    \subsection{Módulo de análisis}
    \begin{enumerate}[label=RF\arabic*.]
    	\setcounter{enumi}{11}
    	\item Mecanismo para el tratamiento de la imagen previo a su análisis.
    	\item Mecanismo para el reconocimiento de un conjunto definido de expresiones matemáticas en imágenes.
    \end{enumerate}
    
    \subsection{Módulo de traducción}
    \begin{enumerate}[label=RF\arabic*.]
    	\setcounter{enumi}{13}
    	\item Mecanismo para la traducción a \LaTeX{} de las expresiones matemáticas encontradas en el modulo de análisis.
    \end{enumerate}
    
\section{Requerimientos No Funcionales} 
    A continuación se enlistan los requerimientos no funcionales.
        \begin{enumerate}[label=RNF\arabic*.]
            \item Escalabilidad. El sistema deberá ser fácilmente escalable con la finalidad de soportar a una mayor cantidad de usuarios permitiendo dar respuestas menores a \textbf{x minutos} por petición de traducción.

            \item Disponibilidad. El sistema debería estar disponible en la mayor parte del tiempo a excepción de operaciones de soporte planeadas y previamente notificadas al usuario final.
    
            \item Estabilidad.
            
            \item Seguridad.
            
            
        \end{enumerate}    
\section{Requerimientos técnicos}
    A continuación se en listan los requerimientos técnicos para un mejor funcionamiento del sistema:
    \subsection{Aplicación móvil}
    \subsubsection{Requerimientos mínimos de software}
    \begin{enumerate}
        \item Sistema operativo Android 4.
    \end{enumerate}
    
    \subsubsection{Requerimientos mínimos de hardware}
    \begin{enumerate}
        %\item Resolución cámara: 13 Megapixeles
        %\item Procesador: Dualcore de 1.2 GHz.
        \item Memoria RAM: 2 GB.
        \item Espacio de almacenamiento de 50 MB.
    \end{enumerate}
    \subsection{Aplicación web}
    \subsubsection{Requerimientos mínimos de software}
    \begin{enumerate}
        \item Cualquiera de los siguientes navegadores web:
        \begin{itemize}
            \item Google Chrome 7
            \item Edge
            \item Internet Explorer 10
            \item Firefox 4
            \item Opera 12
            \item Safari 5
        \end{itemize}
    \end{enumerate}
    \subsubsection{Requerimientos mínimos de hardware}
    \begin{enumerate}
        \item Memoria RAM: 2 GB.
        \item Espacio de almacenamiento de 100 MB.
    \end{enumerate}
    %Poner los recomendados?
\section{Reglas de Negocio}
En esta sección se presentan las reglas de negocio que se necesitan para la elaboración del sistema.
\subsection{RN-001 Campos obligatorios} \label{RN001}
    \begin{enumerate}[label= ]
        %\item Tipo: Derivación. 
        %\item Nivel: Controla la operación.
        \item \textbf{Descripción:} Aquellos campos que son obligatorios no pueden dejarse vacíos.
        \item \textbf{Ejemplo:} Si se tiene un formulario donde existan los siguientes campos:
        \begin{itemize}
            \item Campo 1
            \item Campo 2 (obligatorio)
            \item Campo 3
        \end{itemize}
        El usuario puede dejar los campos 1 y 2 vacíos pero el obligatorio no se podrá omitir.
    \end{enumerate}
    
\subsection{RN-002 Datos correctos} \label{RN002}
    \begin{enumerate}[label= ]
        %\item Tipo: Derivación. 
        %\item Nivel: Controla la operación.
        \item \textbf{Descripción:} Para que los datos sean considerados correctos deben de cumplir con lo establecido en el modelo de información del sistema.
    \end{enumerate}

\subsection{RN-003 Unicidad de identificadores} \label{RN003}
    \begin{enumerate}[label= ]
        %\item Tipo: Restricción. 
        %\item Nivel: Controla la operación.
        \item \textbf{Descripción:} En el conjunto de entidades del sistema, no puede existir elementos con el mismo identificador.
        \item \textbf{Ejemplo de cumplimiento:} Registro de usuario en el cual el correo es el identificador y se puede dar el siguiente caso.
              \begin{itemize}
                  \item Usuario1: \{nombre=Carlos, correo=carlos\_isc7@outlook.com\}
                  \item Usuario2: \{nombre=Juan, correo=gladwell45@outlook.com”\}
              \end{itemize}
        \item \textbf{Ejemplo de fallo:} Registro de usuario en el cual el correo es el identificador y por ende no se puede dar el siguiente caso.
            \begin{itemize}
                \item Usuario1: \{nombre=Ian, correo=carlos\_isc7@outlook.com\}
                \item  Usuario2: \{nombre=Juan, correo=carlos\_isc7@outlook.com\}
            \end{itemize}
    \end{enumerate}

\subsection{RN-004 Calificación proyecto} \label{RN004}
    \begin{enumerate}[label= ]
        %\item Tipo:  
        %\item Nivel: Controla la operación.
        \item \textbf{Descripción:} La calificación de un proyecto tendrá un valor entre 1 y 5 resultado del promedio de las traducciones asociadas a dicho proyecto sin decimales que se hará a través de un redondeo.
        \item \textbf{Sentencia:} Sea: \\
        \[ C_p = \frac{1}{n} \sum_{i=1}^{n} x_i \]
        Donde $C_p$ es la calificación del proyecto, $x_i$ es la calificación de la $i$ traducción y $n$ es el número de traducciones calificadas en el proyecto, deberá actualizarse cada que se agregue o elimine una traducción calificada.
         \item \textbf{Ejemplo:} Si se tiene un proyecto con cinco traducción pero solo tres de ellas tienen calificación se tiene lo siguiente:
         \begin{itemize}
             \item Traducción 1: calificación=4
             \item Traducción 2: Sin calificación
             \item Traducción 3: Sin calificación
             \item Traducción 4: calificación=2
             \item Traducción 5: calificación=5
         \end{itemize}
         La calificación del proyecto sera:
           \[ C_p= \frac{4+3+5}{3} = 4 \]
    \end{enumerate}

\subsection{RN-005 Fecha de modificación} \label{RN005}
    \begin{enumerate}[label= ]
        %\item Tipo: . 
        %\item Nivel: Controla la operación.
        \item \textbf{Descripción:} La fecha de modificación se actualizará cada que se agregue o modifique una traducción al proyecto o que se modifique el proyecto en si la fecha se actualizará con el valor al momento en que se haga la modificación.
       \item \textbf{Ejemplo}. Si se tiene un proyecto con la fecha 2019-05-03 y se realiza una modificación el día 2019-05-20 ahora la fecha de modificación del proyecto será 2019-05-20.
    \end{enumerate}
    
\subsection{RN-006 Usuario verificado} \label{RN006}
    \begin{enumerate}[label= ]
        %\item Tipo: Controladora (Batiz puso esta en una similar) 
        %\item Nivel: Controla la operación.
        \item \textbf{Descripción:} Para que un usuario pueda acceder al sistema su cuenta debe de estar verificada.
    \end{enumerate}
\subsection{RN-007 Información necesaria para descargar un proyecto} \label{RN007}
    \begin{enumerate}[label= ]
        %\item Tipo: Controladora (Batiz puso esta en una similar) 
        %\item Nivel: Controla la operación.
        \item \textbf{Descripción:} Para poder descargar un proyecto es necesario que este cuente con un nombre y al menos una traducción realizada.
    \end{enumerate}
\section{Análisis de Riesgos}
Identificar, clasificar y analizar los riesgos potenciales para el presente Trabajo Terminal, nos permite prevenir las posibles amenazas y probables eventos no deseados así como los daños y consecuencias que estos puedan traer al proyecto. La Tabla 

\begin{center}
    \begin{longtable}{|J{2.5cm}|J{2cm}|J{5cm}|J{5cm}|}
    
        \hline
        \begin{center}
            \textbf{Área de impacto}
        \end{center}                &
        \begin{center}
            \textbf{Nivel de impacto}
        \end{center}                &
        \begin{center}
            \textbf{Causas} 
        \end{center}                &
        \begin{center}
            \textbf{Métodos para contrarrestar el riesgo}
        \end{center} \\ 
        
        \hline
        Funcionalidad de la aplicación.         &
        \begin{center}
            Alto
        \end{center}                            &
        \begin{itemize}
            \item No obtener un conjunto de entrenamiento lo suficientemente grande para realizar una traducción con pocos errores.
            \item Poca precisión por parte del modelo utilizado.
        \end{itemize}                           &
        Investigar sobre posibles nuevos conjuntos de entrenamiento que la comunidad científica libere. \\
        
        \hline
        Software                                &
        \begin{center}
            Medio
        \end{center}                            &
        \begin{itemize}
            \item Cambios drásticos en las herramientas de desarrollo establecidas para el desarrollo del sistema.
        \end{itemize}                           &
        Seleccionar tecnologías que sean estables y utilizadas en la industria. \\
        
        \hline
        Competitividad de la aplicación         &
        \begin{center}
            Alto
        \end{center}                            &
        \begin{itemize}
            \item El lanzamiento de una aplicación similar por parte de una compañía mucho más grande.
        \end{itemize}                           &
        Seguir incrementando la precisión así como las capacidades del presente proyecto con el fin de que pueda competir en el mercado. \\
        
        \hline
        Confianza del usuario                   &
        \begin{center}
            Alto
        \end{center}                            &
        \begin{itemize}
            \item Insatisfacción por parte del usuario por factores tales como la poca precisión de la traducción.
        \end{itemize}                           &
        Seguir incrementando la precisión. Pedirle retroalimentación directa al usuario final. \\
        
        \hline
        
        \caption{Análisis de riesgos.}
        \label{tbl:analisis-riesgos}
        
    \end{longtable}
\end{center}
\newpage   
\section{Descripción del software}
\subsection{Android}
Android es un sistema operativo para móviles que en la actualidad es desarrollado por Google. Esta principalmente pensado para dispositivos con pantalla táctil como smartphones y tablets. Y que en la actualidad junto con iOS son las principales opciones en cuanto a sistemas operativos para teléfonos móviles. Es por esto que en la Tabla \ref{tbl:comparativa-moviles} se hace una comparativa de las principales características que se tomaron en cuenta para la elección de sistema operativo de dispositivos móviles que se usaría en este trabajo.
\begin{center}
    \begin{longtable}{|J{3cm}|J{5cm}|J{5cm}|}
    \hline
    \textbf{Características} & \textbf{Android} & \textbf{iOS} \\ \hline
    Núcleo & UNIX & UNIX \\ \hline
    Lenguajes de desarrollo & Java, C, C++, Kotlin & Swift, C, C++, Objective-C \\ \hline
    Ambiente de desarrollo & Android Studio, disponible en Windows, Linux y MacOS & Xcode, solo disponible en MacOS \\ \hline
    Complejidad de desarrollo & Medio complejo debido a la gran cantidad de diferencias entre dispositivos y versiones de sistema operativo & Poco compleja debido a que hay poca diversidad de versiones sistemas operativos y dispositivos \\ \hline
    Tiempo de desarrollo & Suele tomar 30\%-40\% más que para iOS \cite{ddi} & Depende de la complejidad del desarrollo \\ \hline
    Tiempo de despliegue & Es rápido desplegar una aplicación de Android debido a los test automáticos que se realizan sobre esta para su verificación & Es lento su despliegue debido a que la verificación es manual \\ \hline
    Cantidad del mercado & La cantidad de usuarios es mayor, tan solo el primer cuarto del 2017 el 86.1\% de los teléfonos vendidos fueron Android \cite{ddi} & Su mercado es menor al de Android \\ \hline
    Código abierto & El kernel, UI y algunas aplicaciones estándar son de código abierto & El kernel no es de código abierto pero esta basado en Darwin OS que es de código abierto \\ \hline
    Ultima versión & Android 10 (Septiembre 3, 2019) & iOS 13 (Septiembre 19, 2019) \\ \hline
    Seguridad & Actualizaciones de seguridad bastante regulares. Sin embargo, debido a los fabricantes dichas actualizaciones pueden demorarse en ser aplicadas. También se pueden instalar aplicaciones externas a la Play Store por lo que esto puede generar problemas de seguridad & Pocas actualizaciones, las amenazas de seguridad son pocas debido a que descargar aplicaciones fuera de la App Store es complicado. \\ \hline
    \caption{Tabla comparativa de sistemas operativos de dispositivos móviles}
    \label{tbl:comparativa-moviles}
    \end{longtable}
\end{center}

Android nos brinda un ambiente de desarrollo más flexible que el que presenta iOS debido a las características listadas con anterioridad. Es cierto que el desarrollo puede ser más tardado sin embargo para este proyecto el optar por iOS nos generaría problemas debido a que no solo la plataforma de desarrollo es más cerrada sino que el tiempo de desarrollo seria mayor debido a la curva de aprendizaje de Swift y Objective-C a la cual nos enfrentaríamos.

Finalmente, la cantidad del mercado que tiene Android es significativamente mayor que la que tiene iOS por ende este es un gran punto a considerar. Es por todas estas cuestiones que se opto por desarrollar el trabajo para la plataforma de Android.
\subsection{Base de datos}

Un Sistema Gestor de Bases de Datos (SGBD) o DGBA (Data Base Management System) es un conjunto de programas no visibles que administran y gestionan la información que contiene una base de datos. Para el caso del presente Trabajo Terminal, se decidio utilizar una base de datos relacional. La Tabla \ref{tbl:analisis-bases} muestra una comparativa de los tres principales gestores de bases de datos \cite{basesComparacion}.

\begin{center}
    \begin{longtable}{|m{3.5cm}|m{3.5cm}|m{3.5cm}|m{3.5cm}|}
    
    \hline
    \begin{center}
        \textbf{Características}
    \end{center}                &
    \begin{center}
        \textbf{MySQL}
    \end{center}                &
    \begin{center}
        \textbf{Oracle}
    \end{center}                &
    \begin{center}
        \textbf{PostgreSQL}
    \end{center} \\
    
    \hline
    Modelo de base de datos primario    &
    Relacional                          &
    Relacional                          &
    Relacional  \\
    
    \hline
    Modelo de base de datos secundario  &
    Documento                           &
    Documento, gráfo, RDF               &
    Documento \\
    
    \hline
    Distribución                        &
    Código abierto                      &
    Comercial                           &
    Código abierto \\
    
    \hline
    Implementación                      &
    C y C++                             &
    C y C++                             &
    C \\
    
    \hline
    Sistemas operativos soportados           &
    FreeBSD, Linux, OS X, Solaris y Windows. &
    AIX, HP-UX, Linux, OS X, Solaris, 
    Windows y z/OS.                          &
    FreeBSD, HP-UX, Linux, NetBSD, 
    OpenBSD, OS X, Solaris, Unix, Windows. \\
    
    \hline
    Soporte de XML                      &
    Si                                  &
    Si                                  &
    Si \\
    
    \hline
    Scripts del lado del servidor       &
    Si                                  &
    PL/SQL                              &
    Funciones definidas por el usuario. \\
    
    \hline
    Triggers                            &
    Si                                  &
    Si                                  &
    Si \\
    
    \hline
    Transacciones                       &
    ACID                                &
    ACID                                &
    ACID \\
    
    \hline
    
    \caption{Comparación de diferentes gestores de bases de datos.}
    \label{tbl:analisis-bases}
    
    \end{longtable}
\end{center}

Para el desarrollo del presente proyecto, se selecciono PostgreSQL como gestor de bases de datos, debido a que es de código abierto y contiene más funcionalidades avanzadas de lo que MySQL soporta.




\subsection{Framework de desarrollo web}

Un framework de desarrollo es un conjunto de utilerías y funciones que permiten de una manera consistente acelerar el proceso de creación de una aplicación web. La Tabla \ref{tbl:analisis-frameworks} muestra una comparativa entre los principales frameworks en la industria \cite{fComparacion}.

\begin{center}
    \begin{longtable}{|m{3.5cm}|m{3.5cm}|m{3.5cm}|m{3.5cm}|}
    
    \hline
    \begin{center}
        \textbf{Características}
    \end{center}                &
    \begin{center}
        \textbf{Django}
    \end{center}                &
    \begin{center}
        \textbf{Ruby on Rails}
    \end{center}                &
    \begin{center}
        \textbf{Laravel}
    \end{center} \\
    
    \hline
    Lenguaje soportado          &
    Python                      &
    Ruby                        &
    PHP \\
    
    \hline
    Mercado                     &
    1,192 compañías lo utilizan &
    2,723 compañías lo utilizan &
    1,032 compañías lo utilizan \\
    
    \hline
    Ecosistema                  &
    Muchos paquetes disponibles, entre las más importantes se encuentran: Django REST Framework, Django allauth para la autenticación con redes sociales y Celery.  &
    Cientos de gemas disponibles, librerias como Action Mailer y Active Storage, frameworks como Active Job y Action Cable. &
    Se compone de mas de 15k paquetes. Los más populares son: Cashier, Envoy and Passport, Scout, Socialite, Envoyer, Forge, Horizon, Lumen, entre otros. \\
    
    \hline
    Desempeño                   &
    Respuestas rápidas de texto plano, actualizaciones de la BD eficientes, excelente en la serialización de JSON.      &
    Respuestas buenas de texto plano, eficiente en las actualizaciones de la DB, eficiente respuesta en JSON.  &
    Respuestas buenas de texto plano, excelente en las actualizaciones de la DB. \\
    
    \hline
    Seguridad                                           &
    Provee métodos contra inyección SQL, ataques XSS, CSRF y \textit{clickjacking}. Excelente en el manejo de autenticación de usuarios y permisos. Actualizaciones constantes. &
    Provee métodos contra inyección SQL, ataques XSS, CSRF y \textit{clickjacking}.                              &
    Es vulnerable a ataques, provee métodos de autenticación y hashing. \\
    
    \hline
    \caption{Comparación de frameworks de desarrollo web.}
    \label{tbl:analisis-frameworks}
    
    \end{longtable}
\end{center}

Para el desarrollo de la parte web del presente proyecto, se eligió Django como framework de desarrollo debido a que esta enteramente desarrollado en python, lo que nos permite el uso de las mejores librerias de deep learning, las cuales solo estan soportadas en este lenguaje. Además, Django provee de un conjunto muy robusto de métodos para mantener segura la información.

