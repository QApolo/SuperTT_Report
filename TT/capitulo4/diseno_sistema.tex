\chapter{Diseño del sistema}
%Toda la documentación del sistema hay varias cosas que pueden ir aqui
%
%\begin{itemize}
%    \item Arquitectura del sistema
%    \item Diagrama de clases
%    \item Base de datos
%    \item Diagrama de paquetes
%    \item Diagrama de procesos
%    \item Casos de uso
%    \item Interfaces
%    \item Diseño de la arquitectura de la red neuronal
%\end{itemize}
%No se si tengamos que hacer todos pero al menos la arquitectura del sistema, de la red, casos de uso, interfaces y base de datos yo diría que sí



\section{Arquitectura del sistema}
\begin{figure}[h]
    \centering
    \includegraphics[width=\textwidth]{capitulo4/imagenes/ArquitecturaApp.jpg}
    \caption{Arquitectura del sistema.}
    \label{fig:arquitectura}
\end{figure}

\section{Modelo de datos del sistema}
En la figura \ref{fig:db} se aprecia el modelo de datos del sistema y se describen las entidades que lo conforman, así como los respectivos atributos de cada una de estas entidades.
\begin{figure}[h]
	\centering
	\includegraphics[width=\textwidth]{capitulo4/imagenes/tt_base.png}
	\caption{Modelo de datos del sistema.}
	\label{fig:db}
\end{figure}

\subsection{Entidad: Usuario}
Se refiere a la cuenta que un usuario puede tener, es la forma en la que accede al sistema.
\subsubsection{Atributos}
\begin{center}
	\begin{longtable}{|J{3cm}|J{4cm}|J{3cm}|J{2cm}|J{2cm}|}
		\hline
		\textbf{Nombre} & \textbf{Descripción} & \textbf{Tipo} & \textbf{Requerido} & \textbf{Único} \\ \hline
		id & & & Sí & Sí\\ \hline
		nombre & & & Sí & No \\ \hline
		apellidos & & & Sí & No \\ \hline
		email & & & Sí & Sí\\ \hline
		imagen\_perfil & & & Sí & No \\ \hline
		activo & & & Sí & No \\ \hline
		id\_estudios & & & No & No \\ \hline
		password & & & Sí & No \\ \hline
		\caption{Tabla de los atributos de la entidad usuario}
		\label{tbl:entidad-usuario}
	\end{longtable}
\end{center}
\subsection{Entidad: Proyecto}
Se refiere a los proyectos que están asociados a un usuario, un proyecto esta compuesto por varias traducciones.
\subsubsection{Atributos}
\begin{center}
	\begin{longtable}{|J{3cm}|J{4cm}|J{3cm}|J{2cm}|J{2cm}|}
		\hline
		\textbf{Nombre} & \textbf{Descripción} & \textbf{Tipo} & \textbf{Requerido} & \textbf{Único} \\ \hline
		id & & & Sí & Sí \\ \hline
		usuario\_id & & & Sí & No \\ \hline
		nombre & & & Sí & No \\ \hline
		fechaModificacion & & & Sí & No \\ \hline
		fechaCreacion & & & Sí & No \\ \hline
		calificacion & & & Sí & No \\ \hline
		\caption{Tabla de los atributos de la entidad proyecto}
		\label{tbl:entidad-proyecto}
	\end{longtable}
\end{center}
\subsection{Entidad: Traducción}
Hace referencia a la traducción de \LaTeX que forma parte de un proyecto y que pertenece a un usuario. Esta traducción es el resultado de analizar una imagen.
\subsubsection{Atributos}
\begin{center}
	\begin{longtable}{|J{3cm}|J{4cm}|J{3cm}|J{2cm}|J{2cm}|}
		\hline
		\textbf{Nombre} & \textbf{Descripción} & \textbf{Tipo} & \textbf{Requerido} & \textbf{Único} \\ \hline
		id & & & Sí & Sí \\ \hline
		proyecto\_id & & & Sí & No \\ \hline
		usuario\_id & & & Sí & No \\ \hline
		nombre & & & Sí & No \\ \hline
		calificacion & & & Sí & No\\ \hline
		fechaCreacion & & & Sí & No \\ \hline
		\caption{Tabla de los atributos de la entidad traducción}
		\label{tbl:entidad-traduccion}
	\end{longtable}
\end{center}
\subsection{Entidad: Tipo estudios}
Hace referencia a los distintos grados de estudios que tiene un usuario.
\subsubsection{Atributos}
\begin{center}
	\begin{longtable}{|J{3cm}|J{4cm}|J{3cm}|J{2cm}|J{2cm}|}
		\hline
		\textbf{Nombre} & \textbf{Descripción} & \textbf{Tipo} & \textbf{Requerido} & \textbf{Único} \\ \hline
		id & & & Sí & Sí \\ \hline
		grado & & & Sí & No \\ \hline
		\caption{Tabla de los atributos de la entidad tipo de estudios}
		\label{tbl:entidad-tipo-estudios}
	\end{longtable}
\end{center}
\subsection{Entidad: Usuario Token}
Se refiere al token de autenticación que tiene un usuario al momento de crear una cuenta y que su utiliza en la comunicación entre la aplicación web y android.
\subsubsection{Atributos}
\begin{center}
	\begin{longtable}{|J{3cm}|J{4cm}|J{3cm}|J{2cm}|J{2cm}|}
		\hline
		\textbf{Nombre} & \textbf{Descripción} & \textbf{Tipo} & \textbf{Requerido} & \textbf{Único} \\ \hline
		token & & & Sí &  Sí\\ \hline
		created & & & Sí & No\\ \hline
		user\_id & & & Sí & No\\ \hline
		\caption{Tabla de los atributos de la entidad usuario token}
		\label{tbl:entidad-usuario-token}
	\end{longtable}
\end{center}

\section{Aplicación de Android}
\subsection{Módulo usuarios}
\subsubsection{Diagrama de casos de uso}
En la figura \ref{fig:modulo-usuarios-web} se observa el diagrama de casos de uso correspondiente a este módulo.

\begin{figure}[H]
    \centering
    \includegraphics[width=400px]{capitulo4/web/modulo-usuarios-web.jpg}
    \caption{Diagrama de casos de uso del módulo Usuarios Web}
    \label{fig:modulo-usuarios-web}
\end{figure}

\input{capitulo4/web/usuarios/w-usr-cu1}
\newpage
\subsubsection{W-USR-CU1.1 Verificar cuenta}
Este caso de uso permite al usuario verificar su cuenta en el sistema.

\begin{longtable}{|J{5cm}|J{10.3cm}|}
	\hline
	\textbf{Nombre del Caso de Uso} &
		W-USR-CU1.1 Verificar cuenta \\ \hline
	\textbf{Objetivo} &
		Completar el registro de un usuario mediante la verificación de su cuenta \\ \hline
	\textbf{Actores} &
		Usuario no verificado \\ \hline 
	\textbf{Entradas} & 
		Ninguna \\ \hline 
	\textbf{Salidas} & 
		Ninguna \\ \hline
	\textbf{Precondiciones} &
		\begin{itemize}
		    \item El actor debe estar registrado en el sistema con un estatus no verificado
		\end{itemize} \\ \hline
	\textbf{Postcondiciones} &
	        La cuenta del actor será registrada con el estatus de verificada \\ \hline
	\textbf{Prioridad} & 
		Alta. \\ \hline
	\textbf{Reglas de Negocio} & 
		Ninguna \\ \hline

	% \caption{}
	%\label{desc:SUB-M-CU1}
\end{longtable}

\paragraph{Trayectoria Principal}
\label{W-USR-CU1.1}
	\begin{enumerate}
	    \item {[Sistema]} Verifica el código de autenticación.
	    
	    \item {[Sistema]} Cambia el estatus del actor como \textit{Verificado}.
	    
	    \item {[Sistema]} Muestra el mensaje \hyperref[MSG9]{MS9 Operación exitosa} en la pantalla \textbf{\nameref{fig:MIUW-1}}.
	\end{enumerate}
	Fin del caso de uso.
\newpage
\subsubsection{W-USR-CU2 Iniciar sesión}
Para poder hacer uso toda la funcionalidad de la aplicación es necesario que el usuario inicie sesión.
\begin{longtable}{|J{5cm}|J{10.3cm}|}
	\hline
	\textbf{Nombre del Caso de Uso} &
		W-USR-CU2 Iniciar sesión \\ \hline
	\textbf{Objetivo} &
		Permitir a un usuario validado y con cuenta validada el ingreso al sistema \\ \hline
	\textbf{Actores} &
		Usuario verificado \\ \hline 
	\textbf{Disparador} & 
		El actor requiere iniciar sesión \\ \hline 
	\textbf{Entradas} & 
		\begin{itemize}
				\item Correo
				\item Contraseña
		\end{itemize}\\ \hline 
	\textbf{Salidas} & 
		Ninguna. \\ \hline
	\textbf{Precondiciones} &
		\begin{itemize}
				\item El actor debe de tener una cuenta verificada
		\end{itemize} \\ \hline
	\textbf{Postcondiciones} &
		\begin{itemize}
			\item El actor pude hacer uso del resto de funcionalidades del sistema
		\end{itemize}\\ \hline
	\textbf{Prioridad} & 
		Media \\ \hline
	\textbf{Reglas de Negocio} & 
		\begin{itemize}
			\item \hyperref[RN001]{RN-001 Campos obligatorios}
			\item \hyperref[RN002]{RN-002 Datos correctos}
			\item \hyperref[RN006]{RN-006 Usuario verificado}
		\end{itemize} \\ \hline

	% \caption{}
	%\label{desc:SUB-M-CU1}
\end{longtable}
% Me faltan las referencias chidas
\paragraph{Trayectoria Principal}
	\begin{enumerate}
	    \item {[Actor]} Ingresa al sitio web.
	    \item {[Sistema]} Muestra la pantalla \hyperref[fig:MIUW-1]{\bf MIUW 1 Iniciar Sesión}.
	    \item {[Actor]} Ingresa la información solicitada en pantalla.
	    \item {[Actor]} Presiona el botón \textit{Iniciar sesión}.
	    \item {[Sistema]} Valida la información según la regla de negocio \hyperref[RN001]{RN-001} y \hyperref[RN002]{RN-002}. \hyperref[W-USR-CU2:TA]{[Trayectoria alternativa A]}
	    \item {[Sistema]} Autentica al usuario según la regla de negocio \hyperref[RN006]{RN-006}. \hyperref[W-USR-CU2:TB]{[Trayectoria alternativa B]}
	    \item {[Sistema]} Muestra la pantalla \hyperref[fig:MIUW-5]{\bf MIUW 5 Lista de Proyectos}.
	\end{enumerate}
	Fin del caso de uso.

\paragraph{Trayectoria Alternativa A} \label{W-USR-CU2:TA}
	La información ingresada por el actor no es valida.
	\begin{enumerate}[label=A\arabic*.]
		\item {[Sistema]} Muestra el mensaje \hyperref[MSG1]{\bf MSG1 Datos no validos} en la pantalla \hyperref[fig:MIUW-1]{\bf MIUW 1 Iniciar Sesión}.
	\end{enumerate}
	Fin de la trayectoria alternativa.
\paragraph{Trayectoria Alternativa B} \label{W-USR-CU2:TB}
	El usuario no ha verificado su cuenta.
	\begin{enumerate}[label=B\arabic*.]
		\item {[Sistema]} Muestra el mensaje \hyperref[MSG2]{\bf MSG2 Cuenta no verificada} en la pantalla \hyperref[fig:MIUW-1]{\bf{MIUW 1 Iniciar Sesión}}.
	\end{enumerate}
	Fin de la trayectoria alternativa.
\newpage
\input{capitulo4/web/usuarios/w-usr-cu3}
\newpage
\subsubsection{W-USR-CU4 Editar perfil}
La posibilidad de modificar los datos de una cuenta es indispensable para un usuario por lo que se proporciona un mecanismo para realizar esta tarea.
\begin{longtable}{|J{5cm}|J{10.3cm}|}
	\hline
	\textbf{Nombre del Caso de Uso} &
		W-USR-CU4 Editar perfil \\ \hline
	\textbf{Objetivo} &
		Permitir al actor editar los datos de su cuenta. \\ \hline
	\textbf{Actores} &
		Usuario verificado \\ \hline 
	\textbf{Disparador} & 
		El actor requiere modificar su información \\ \hline 
	\textbf{Entradas} & 
		\begin{itemize}
		    \item Nombre
		    \item Contraseña
		    \item Confirmación de contraseña
		    \item Grado de estudio
		\end{itemize} \\ \hline 
	\textbf{Salidas} & 
		\begin{itemize}
		    \item \hyperref[MSG9]{\bf MSG9 Operación exitosa}
		\end{itemize} \\ \hline
	\textbf{Precondiciones} &
		\begin{itemize}
				\item El actor debe de haber iniciado sesión
		\end{itemize} \\ \hline
	\textbf{Postcondiciones} &
		\begin{itemize}
			\item El actor modifica la información de su cuenta
		\end{itemize}\\ \hline
	\textbf{Prioridad} & 
		Baja \\ \hline
	\textbf{Reglas de Negocio} & 
		\begin{itemize}
		    \item \hyperref[RN001]{RN-001 Campos obligatorios}
		    \item \hyperref[RN002]{RN-002 Datos correctos}
		\end{itemize} \\ \hline

	% \caption{}
	%\label{desc:SUB-M-CU1}
\end{longtable}
% Me faltan las referencias chidas
\paragraph{Trayectoria Principal}
	\begin{enumerate}
	    \item {[Actor]} Presiona el la opción \textit{Cuenta} del menú \hyperref[fig:MIUW-7]{\bf MIUW 7 Menú de Opciones}.
	    \item {[Sistema]} Muestra la pantalla  \hyperref[fig:MIUW-3]{\bf MIUW 3 Editar Perfil}.
	    \item {[Actor]} Modifica la información.
	    \item {[Actor]} Presiona el botón guardar.
	    \item {[Sistema]} Valida la información según la regla de negocio \hyperref[RN001]{RN-001} y \hyperref[RN002]{RN-002}. \hyperref[W-USR-CU4:TA]{[Trayectoria Alternativa A]}
	    \item {[Sistema]} Persiste la información modificada.
	    \item {[Sistema]} Muestra el mensaje \hyperref[MSG9]{\bf MSG9 Operación exitosa} en la pantalla \hyperref[fig:MIUW-3]{\bf MIUW 3 Editar Perfil}.
	\end{enumerate}
	Fin del caso de uso.

\paragraph{Trayectoria Alternativa A} \label{W-USR-CU4:TA}
	Datos incorrectos.
	\begin{enumerate}[label=A\arabic*.]
		\item {[Sistema]} Muestra el mensaje \hyperref[MSG1]{\bf MSG11 Datos no validos} en la pantalla \hyperref[fig:MIUW-3]{\bf MIUW 3 Editar Perfil}.
	\end{enumerate}
	Fin de la trayectoria alternativa.
\newpage
\subsubsection{W-USR-CU5 Cerrar sesión}
Este caso de uso permite al usuario final cerrar su sesión en la aplicación.

\begin{longtable}{|J{5cm}|J{10.3cm}|}
	\hline
	\textbf{Nombre del Caso de Uso} &
		W-USR-CU5 Cerrar sesión \\ \hline
	\textbf{Objetivo} &
		Cerrar sesión en la aplicación \\ \hline
	\textbf{Actores} &
		Usuario verificado \\ \hline 
	\textbf{Disparador} & 
		El actor requiere cerrar su sesión \\ \hline 
	\textbf{Entradas} & 
		Ninguna. \\ \hline 
	\textbf{Salidas} & 
		Ninguna. \\ \hline
	\textbf{Precondiciones} &
		\begin{itemize}
		    \item El actor debe estar registrado en el sistema con un estatus de verificado
		    \item El actor debe haber iniciado sesión
		\end{itemize} \\ \hline
	\textbf{Postcondiciones} &
	        Se cierra la sesión \\ \hline
	\textbf{Prioridad} & 
		Media \\ \hline
	\textbf{Reglas de Negocio} & 
		Ninguna \\ \hline

	% \caption{}
	%\label{desc:SUB-M-CU1}
\end{longtable}

\paragraph{Trayectoria Principal}
\label{W-USR-CU5}
	\begin{enumerate}
	    \item {[Actor]} Solicita cerrar sesión oprimiendo el botón \textit{Cerrar Sesión} del menú \textbf{\nameref{fig:MIUW-7}}.
	    
	    \item {[Sistema]} Cierra la sesión del actor en el servidor.
	    
	    \item {[Sistema]} Muestra la pantalla \textbf{\nameref{fig:MIUW-1}}.
	\end{enumerate}
	Fin del caso de uso.

\newpage

\subsection{Módulo proyectos}
\subsubsection{Diagrama de casos de uso}
En la figura \ref{fig:modulo-proyectos-web} se observa el diagrama de casos de uso correspondiente a este módulo.

\begin{figure}[H]
    \centering
    \includegraphics[width=400px]{capitulo4/web/modulo-proyectos-web.jpg}
    \caption{Diagrama de casos de uso del módulo Proyectos Web}
    \label{fig:modulo-proyectos-web}
\end{figure}

\input{capitulo4/web/proyectos/w-pr-cu1}
\newpage
\input{capitulo4/web/proyectos/w-pr-cu1.1}
\newpage
\subsubsection{W-PR-CU1.1.1 Editar proyecto}
Este caso de uso permite al actor cambiar el nombre de algún proyecto previamente creado por él.

\begin{longtable}{|J{5cm}|J{10.3cm}|}
	\hline
	\textbf{Nombre del Caso de Uso} &
		W-PR-CU1.1.1 Editar proyecto \\ \hline
	\textbf{Objetivo} &
		Cambiar la información de algún proyecto \\ \hline
	\textbf{Actores} &
		Usuario verificado \\ \hline 
	\textbf{Disparador} & 
		El actor requiere cambiar la información de algún proyecto. \\ \hline 
	\textbf{Entradas} & 
		\begin{itemize}
				\item Nombre del proyecto
		\end{itemize}\\ \hline 
	\textbf{Salidas} & 
		\begin{itemize}
			\item \hyperref[MSG9]{\bf MSG9 Operación exitosa}
		\end{itemize} \\ \hline
	\textbf{Precondiciones} &
		\begin{itemize}
		    \item El actor debe estar registrado en el sistema con un estatus de validado
		    \item El usuario debe haber iniciado sesión
		    \item El proyecto debe de pertenecer al actor
		\end{itemize} \\ \hline
	\textbf{Postcondiciones} &
		Los datos del proyecto se modifican \\ \hline
	\textbf{Prioridad} & 
		Baja \\ \hline
	\textbf{Reglas de Negocio} & 
		\begin{itemize}
    		\item \hyperref[RN001]{RN-001 Campos obligatorios}
    		\item \hyperref[RN002]{RN-002 Datos correctos} 
    		\item \hyperref[RN005]{RN-005 Fecha de modificación}
    	\end{itemize}\\ \hline

	% \caption{}
	%\label{desc:SUB-M-CU1}
\end{longtable}

\paragraph{Trayectoria Principal}
\label{W-PR-CU1.1.1}
	\begin{enumerate}
	    \item {[Actor]} Presiona el botón \textit{Editar} en la pantalla \textbf{\nameref{fig:MIUW-6}}.
	    
	    \item {[Sistema]} Muestra un campo de texto con el actual nombre del proyecto en la pantalla \textbf{\nameref{fig:MIUW-6}}.
	    
	    \item {[Actor]} Ingresa el nuevo nombre del proyecto.
	    
	    \item {[Actor]} Presiona el botón aceptar \textit{Aceptar} de la pantalla \textbf{\nameref{fig:MIUW-6}}.  \hyperref[W-PR-CU1.1.1:TA]{[Trayectoria alternativa A]}
	    
	    \item {[Sistema]} Verifica los datos como se especifica en las reglas de negocio \hyperref[RN001]{RN-001} y \hyperref[RN002]{RN-002}. \hyperref[W-PR-CU1.1.1:TB]{[Trayectoria alternativa B]}
	    
	    \item {[Sistema]} Calcula la fecha de modificación del proyecto en el sistema según la regla de negocio \hyperref[RN005]{RN-005}.
	    
	    \item {[Sistema]} Persiste los cambios realizados.
	    
	    \item {[Sistema]} Muestra el mensaje \hyperref[MSG9]{\bf MSG9 Operación exitosa} en la pantalla \textbf{\nameref{fig:MIUW-6}}y actualiza la información mostrada. 

	\end{enumerate}
	Fin del caso de uso.
	
\paragraph{Trayectoria Alternativa A} \label{W-PR-CU1.1.1:TA}
	El actor cancela la operación.
	%para más trayectorias, copiar y pegar toda esta sección y en el comando siguiente cambiar por la letra correspondiente
	\begin{enumerate}[label=A\arabic*.]
		\item {[Actor]} Solicita cancelar la operación oprimiendo el botón \textit{Cancelar} en la pantalla \textbf{\nameref{fig:MIUW-6}}.
		\item {[Sistema]} Muestra la pantalla \textbf{\nameref{fig:MIUW-6}}.
	\end{enumerate}
	Fin de la trayectoria alternativa.

\paragraph{Trayectoria Alternativa B} \label{W-PR-CU1.1.1:TB}
	Datos no validos.
	%para más trayectorias, copiar y pegar toda esta sección y en el comando siguiente cambiar por la letra correspondiente
	\begin{enumerate}[label=B\arabic*.]
		\item {[Sistema]} Muestra el mensaje \hyperref[MSG1]{\bf MSG1 Datos no validos} en la pantalla \textbf{\nameref{fig:MIUW-6}}.
		\item {[Actor]} Continua con el paso 3 de la Trayectoria Principal.
	\end{enumerate}
	Fin de la trayectoria alternativa.
\newpage
\subsubsection{W-PR-CU1.1.2 Descargar proyecto}
El poder obtener un archivo de LaTeX para su uso es útil para el usuario por lo que se le brinda esta funcionalidad.

\begin{longtable}{|J{5cm}|J{10.3cm}|}
	\hline
	\textbf{Nombre del Caso de Uso} &
		W-PR-CU1.1.2 Descargar proyecto \\ \hline
	\textbf{Objetivo} &
		Generar un archivo de LaTeX para su descarga  \\ \hline
	\textbf{Actores} &
		Usuario verificado \\ \hline 
	\textbf{Disparador} & 
		El actor requiere descargar las traducciones de un proyecto \\ \hline 
	\textbf{Entradas} & 
		Ninguna \\ \hline 
	\textbf{Salidas} & 
		Archivo de LaTeX \\ \hline
	\textbf{Precondiciones} &
		\begin{itemize}
		    \item El actor debe haber iniciado sesión
		    \item Deben de existir traducciones para descargar el proyecto
		\end{itemize} \\ \hline
	\textbf{Postcondiciones} &
        \begin{itemize}
		    \item Se genera un archivo LaTeX
		\end{itemize} \\ \hline
	\textbf{Prioridad} & 
		Media \\ \hline
	\textbf{Reglas de Negocio} & 
		\begin{itemize}
		    \item \hyperref[RN007]{RN-007 Información necesaria para descargar un proyecto}
		\end{itemize} \\ \hline

	% \caption{}
	%\label{desc:SUB-M-CU1}
\end{longtable}

\paragraph{Trayectoria Principal}
	\begin{enumerate}
	    \item {[Actor]} Presiona el botón \textit{Descargar} de la pantalla  \textbf{\nameref{fig:MIUW-6}}.
	    \item {[Sistema]} Recupera la información necesaria para realizar esta operación según la regla de negocio \hyperref[RN007]{RN-007}. \hyperref[W-PR-CU1.1.2:TA]{[Trayectoria alternativa A]}
	    \item {[Sistema]} Genera un archivo LaTeX con las traducciones del proyecto seleccionado. \hyperref[W-PR-CU1.1.2:TB]{[Trayectoria alternativa B]}
	    \item {[Sistema]} Muestra el mensaje \hyperref[MSG9]{\bf MSG9 Operación exitosa} en la pantalla \textbf{\nameref{fig:MIUW-6}}.
	\end{enumerate}
	Fin del caso de uso.

\paragraph{Trayectoria Alternativa B} \label{W-PR-CU1.1.2:TA}
Falta información para realizar la operación
\begin{enumerate}[label=A\arabic*.]
    \item {[Sistema]} Muestra el mensaje \hyperref[MSG13]{\bf MSG13 Falta información}  en la pantalla \textbf{\nameref{fig:MIUW-6}}.
\end{enumerate}

\paragraph{Trayectoria Alternativa B} \label{W-PR-CU1.1.2:TB}
    No se pudo realizar la operación.
	\begin{enumerate}[label=B\arabic*.]
		\item {[Sistema]} Muestra el mensaje \hyperref[MSG10]{\bf MSG10 Operación fallida}  en la pantalla \textbf{\nameref{fig:MIUW-6}}.
	\end{enumerate}
	Fin de la trayectoria alternativa.
\newpage
\subsubsection{W-PR-CU1.1.3 Eliminar traducción}
Este caso de uso permite al usuario final eliminar una traducción previamente solicitada por él.

\begin{longtable}{|J{5cm}|J{10.3cm}|}
	\hline
	\textbf{Nombre del Caso de Uso} &
		W-PR-CU1.1.3 Eliminar traducción \\ \hline
	\textbf{Objetivo} &
		Eliminar una traducción del usuario \\ \hline
	\textbf{Actores} &
		Usuario verificado \\ \hline 
	\textbf{Disparador} & 
		El actor requiere eliminar una traducción \\ \hline 
	\textbf{Entradas} & 
		Ninguna \\ \hline 
	\textbf{Salidas} & 
		\begin{itemize}
			\item \hyperref[MSG9]{\bf MSG9 Operación exitosa}
		\end{itemize} \\ \hline
	\textbf{Precondiciones} &
		\begin{itemize}
		    \item El actor debe estar registrado en el sistema con un estatus de validado
		    \item El actor debe haber iniciado sesión
		    \item La traducción debe de existir en el sistema
		    \item La traducción debe de pertenecer al actor
		\end{itemize} \\ \hline
	\textbf{Postcondiciones} &
		\begin{itemize}
		    \item Al aceptar la eliminación de una traducción, toda su información correspondiente será eliminada del sistema
		\end{itemize} \\ \hline
	\textbf{Prioridad} & 
		Baja \\ \hline
	\textbf{Reglas de Negocio} & 
		Ninguna \\ \hline

	% \caption{}
	%\label{desc:SUB-M-CU1}
\end{longtable}

\paragraph{Trayectoria Principal}
\label{W-PR-CU1.1.3}
	\begin{enumerate}
	    \item {[Actor]} Presiona el botón \textit{Eliminar} de algún registro mostrado en la pantalla \textbf{\nameref{fig:MIUW-6}}.
	    
	    \item {[Sistema]} Muestra el mensaje \hyperref[MSG11]{\bf MSG11 Confirmación de operación}.
	    
	    \item {[Actor]} Acepta la eliminación oprimiendo el botón \textit{Aceptar}. \hyperref[W-PR-CU1.1.3:TA]{[Trayectoria alternativa A]}
	    
	    \item {[Sistema]} Verifica que la traducción pertenezca al usuario.
	    
	    \item {[Sistema]} Elimina toda la información correspondiente a la traducción.
	    
	    \item {[Sistema]} Muestra el mensaje \hyperref[MSG9]{\bf MSG9 Operación exitosa} en la pantalla \textbf{\nameref{fig:MIUW-6}} indicando que la operación se ha realizado exitosamente y actualiza la lista de traducciones.
	\end{enumerate}
	Fin del caso de uso.
	
\paragraph{Trayectoria Alternativa A} \label{W-PR-CU1.1.3:TA}
	El actor cancela la operación.
	%para más trayectorias, copiar y pegar toda esta sección y en el comando siguiente cambiar por la letra correspondiente
	\begin{enumerate}[label=A\arabic*.]
		\item {[Actor]} Solicita cancelar la operación oprimiendo el botón \textit{Cancelar}.
		\item {[Sistema]} Muestra la pantalla \textbf{\nameref{fig:MIUW-6}}.
	\end{enumerate}
	Fin de la trayectoria alternativa.

\newpage
\subsubsection{W-PR-CU1.1.4 Copiar traducción}
Una traducción que se realice por el sistema puede ser copiada sin necesidad de descargar todo el proyecto sobre el que se trabaja.
\begin{longtable}{|J{5cm}|J{10.3cm}|}
	\hline
	\textbf{Nombre del Caso de Uso} &
		W-PR-CU1.1.4 Copiar traducción \\ \hline
	\textbf{Objetivo} &
		Permitir copiar una traducción realizada  \\ \hline
	\textbf{Actores} &
		Usuario verificado \\ \hline 
	\textbf{Disparador} & 
		El actor requiere copiar una traducción \\ \hline 
	\textbf{Entradas} & 
		Ninguna \\ \hline 
	\textbf{Salidas} & 
		Copia de la traducción \\ \hline
	\textbf{Precondiciones} &
		\begin{itemize}
		    \item El actor debe haber iniciado sesión
		    \item Deben existir traducciones que copiar
		\end{itemize} \\ \hline
	\textbf{Postcondiciones} &
        \begin{itemize}
		    \item Se copia la traducción
		\end{itemize} \\ \hline
	\textbf{Prioridad} & 
		Media \\ \hline
	\textbf{Reglas de Negocio} & 
		Ninguna \\ \hline

	% \caption{}
	%\label{desc:SUB-M-CU1}
\end{longtable}

\paragraph{Trayectoria Principal}
	\begin{enumerate}
	    \item {[Actor]} Presiona el botón \textit{Copiar} de la pantalla  \textbf{\nameref{fig:MIUW-6}}.
	    \item {[Sistema]} Recupera la traducción.
	    \item {[Sistema]} Copia la traducción. \hyperref[W-PR-CU1.1.4:TA]{[Trayectoria alternativa A]}
	    \item {[Sistema]} Muestra el mensaje \hyperref[MSG9]{\bf MSG9 Operación exitosa} en la pantalla \textbf{\nameref{fig:MIUW-6}}.
	\end{enumerate}
	Fin del caso de uso.

\paragraph{Trayectoria Alternativa A} \label{W-PR-CU1.1.4:TA}
    No se pudo realizar la operación.
	\begin{enumerate}[label=A\arabic*.]
		\item {[Sistema]} Muestra el mensaje \hyperref[MSG10]{\bf MSG10 Operación fallida}  en la pantalla \textbf{\nameref{fig:MIUW-6}}.
	\end{enumerate}
	Fin de la trayectoria alternativa.
\newpage
\subsubsection{W-PR-CU1.1.5 Eliminar proyecto}
Este caso de uso permite al usuario final eliminar un proyecto previamente creado por él. 

\begin{longtable}{|J{5cm}|J{10.3cm}|}
	\hline
	\textbf{Nombre del Caso de Uso} &
		W-PR-CU1.1.5 Eliminar proyecto \\ \hline
	\textbf{Objetivo} &
		Eliminar un proyecto del usuario \\ \hline
	\textbf{Actores} &
		Usuario validado \\ \hline 
	\textbf{Disparador} & 
		El actor requiere eliminar un proyecto \\ \hline 
	\textbf{Entradas} & 
		Ninguna \\ \hline 
	\textbf{Salidas} & 
		\begin{itemize}
			\item \hyperref[MSG9]{\bf MSG9 Operación exitosa}
		\end{itemize} \\ \hline
	\textbf{Precondiciones} &
		\begin{itemize}
		    \item El actor debe estar registrado en el sistema con un estatus de verificado
		    \item El actor debe haber iniciado sesión
		    \item El proyecto debe de existir en el sistema
		    \item El proyecto debe de pertenecer al actor
		\end{itemize} \\ \hline
	\textbf{Postcondiciones} &
		\begin{itemize}
		    \item Al aceptar la eliminación del proyecto, todas sus traducciones asociadas serán eliminadas del sistema
		\end{itemize} \\ \hline
	\textbf{Prioridad} & 
		Media \\ \hline
	\textbf{Reglas de Negocio} & 
		Ninguna \\ \hline

	% \caption{}
	%\label{desc:SUB-M-CU1}
\end{longtable}

\paragraph{Trayectoria Principal}
\label{CUW_Eliminar_Proyecto}
	\begin{enumerate}
	    \item {[Actor]} Presiona el botón \textit{Eliminar Proyecto} de la pantalla \textbf{\nameref{fig:MIUW-6}}.
	    
	    \item {[Sistema]} Muestra el mensaje \hyperref[MSG11]{\bf MSG11 Confirmación de operación}.
	    
	    \item {[Actor]} Presiona el botón \textit{Aceptar}. \hyperref[W-PR-CU1.1.5:TA]{[Trayectoria alternativa A]}
	    
	    \item {[Sistema]} Verifica que el proyecto pertenezca al usuario.
	    
	    \item {[Sistema]} Elimina toda la información correspondiente al proyecto.
	    
	    \item {[Sistema]} Muestra el mensaje \hyperref[MSG9]{\bf MSG9 Operación exitosa} en la pantalla \textbf{\nameref{fig:MIUW-5}} y actualiza la lista de proyectos.
	\end{enumerate}
	Fin del caso de uso.
	
\paragraph{Trayectoria Alternativa A} \label{W-PR-CU1.1.5:TA}
	El actor cancela la operación.
	%para más trayectorias, copiar y pegar toda esta sección y en el comando siguiente cambiar por la letra correspondiente
	\begin{enumerate}[label=A\arabic*.]
		\item {[Actor]} Solicita cancelar la operación oprimiendo el botón \textit{Cancelar}.
		\item {[Sistema]} Muestra la pantalla \textbf{\nameref{fig:MIUW-6}}.
	\end{enumerate}
	Fin de la trayectoria alternativa.

\newpage
\subsubsection{W-PR-CU1.1.6.1 Calificar traducción}
Una traducción que se realice por el sistema se puede calificar para apoyar el algoritmo de traducción en un futuro, así como proporcionar retroalimentación al usuario de que tan util es el proyecto que se trabaja.
\begin{longtable}{|J{5cm}|J{10.3cm}|}
	\hline
	\textbf{Nombre del Caso de Uso} &
		W-PR-CU1.1.6.1 Calificar traducción \\ \hline
	\textbf{Objetivo} &
		Permitir la calificación de una traducción  \\ \hline
	\textbf{Actores} &
		Usuario verificado \\ \hline 
	\textbf{Disparador} & 
		El actor requiere calificar una traducción \\ \hline 
	\textbf{Entradas} & 
		\begin{itemize}
		    \item Calificación
		\end{itemize} \\ \hline
	\textbf{Salidas} &
	    \begin{itemize}
		    \item \hyperref[MSG9]{\bf MSG9 Operación exitosa}
		\end{itemize} \\ \hline
	\textbf{Precondiciones} &
		\begin{itemize}
		    \item El actor debe haber iniciado sesión
		    \item Deben existir traducciones que calificar
		\end{itemize} \\ \hline
	\textbf{Postcondiciones} &
        \begin{itemize}
		    \item Se califica la traducción
		    \item Se calcula la calificación del proyecto
		\end{itemize} \\ \hline
	\textbf{Prioridad} & 
		Baja \\ \hline
	\textbf{Reglas de Negocio} & 
		\begin{itemize}
		    \item \hyperref[RN004]{RN-004 Calificación proyecto}
		    \item \hyperref[RN005]{RN-005 Fecha de modificación}
		\end{itemize} \\ \hline

	% \caption{}
	%\label{desc:SUB-M-CU1}
\end{longtable}

\paragraph{Trayectoria Principal}
	\begin{enumerate}
	    \item {[Actor]} Presiona el número de estrellas como calificación
	    \item {[Sistema]} Muestra el mensaje \hyperref[MSG11]{\bf MSG11 Confirmación de operación} en la pantalla \textbf{\nameref{fig:MIUW-6}}.
	    \item {[Actor]} Presiona el botón \textit{Sí}. \hyperref[W-PR-CU1.1.6.1:TA]{[Trayectoria alternativa A]}
	    \item {[Sistema]} Calcula la calificación del proyecto según la regla de negocio \hyperref[RN004]{RN-004}.
	    \item {[Sistema]} Calcula la fecha de modificación según la regla de negocio \hyperref[RN005]{RN-005}.
	    \item {[Sistema]} Persiste la información  \hyperref[W-PR-CU1.1.6.1:TA]{[Trayectoria alternativa B]}
	    \item {[Sistema]} Muestra el mensaje \hyperref[MSG9]{\bf MSG9 Operación exitosa} en la pantalla \textbf{\nameref{fig:MIUW-6}}.
	\end{enumerate}
	Fin del caso de uso.

\paragraph{Trayectoria Alternativa A} \label{W-PR-CU1.1.6.1:TA}
    Se cancela la operación
	\begin{enumerate}[label=A\arabic*.]
	    \item {[Actor]} Presiona el botón \textit{No}.
		\item {[Sistema]} Muestra la pantalla \textbf{\nameref{fig:MIUW-6}}.
	\end{enumerate}
	Fin de la trayectoria alternativa.

\paragraph{Trayectoria Alternativa B} \label{W-PR-CU1.1.6.1:TB}
    No se pudo realizar la operación.
	\begin{enumerate}[label=A\arabic*.]
		\item {[Sistema]} Muestra el mensaje \hyperref[MSG10]{\bf MSG10 Operación fallida}  en la pantalla \textbf{\nameref{fig:MIUW-6}}.
	\end{enumerate}
	Fin de la trayectoria alternativa.
\newpage
\subsubsection{W-PR-CU1.2 Crear proyecto}
Este caso de uso permite al usuario final crear un nuevo proyecto para almacenar sus traducciones.

\begin{longtable}{|J{5cm}|J{10.3cm}|}
	\hline
	\textbf{Nombre del Caso de Uso} &
		W-PR-CU1.2 Crear proyecto \\ \hline
	\textbf{Objetivo} &
		Crear un nuevo proyecto en el cual se puedan almacenar traducciones \\ \hline
	\textbf{Actores} &
		Usuario verificado \\ \hline 
	\textbf{Disparador} & 
		El actor requiere crear un nuevo proyecto \\ \hline 
	\textbf{Entradas} & 
		\begin{itemize}
				\item Nombre del proyecto
		\end{itemize}\\ \hline 
	\textbf{Salidas} & 
		\begin{itemize}
			\item \hyperref[MSG9]{\bf MSG9 Operación exitosa}
		\end{itemize} \\ \hline
	\textbf{Precondiciones} &
		\begin{itemize}
		    \item El actor debe estar registrado en el sistema con un estatus de verificado.
		    \item El usuario debe haber iniciado sesión.
		\end{itemize} \\ \hline
	\textbf{Postcondiciones} &
		Ninguna. \\ \hline
	\textbf{Prioridad} & 
		Media. \\ \hline
	\textbf{Reglas de Negocio} & 
		\begin{itemize}
			\item \hyperref[RN001]{RN-001 Campos obligatorios}
			\item \hyperref[RN002]{RN-002 Datos correctos}
		\end{itemize} \\ \hline

	% \caption{}
	%\label{desc:SUB-M-CU1}
\end{longtable}

\paragraph{Trayectoria Principal}
\label{CUW_Crear_Proyecto}
	\begin{enumerate}
	    \item {[Actor]} Presiona el botón \textit{Crear Proyecto} en la pantalla \textbf{\nameref{fig:MIUW-5}}.
	    
	    \item {[Sistema]} Muestra el mensaje \hyperref[MSG12]{\bf MSG12 Ingrese nombre} en la pantalla \textbf{\nameref{fig:MIUW-5}} solicitando el nombre del nuevo proyecto.
	    
	    \item {[Actor]} Ingresa el nombre del nuevo proyecto.
	    
	    \item {[Actor]} Presiona el botón \textit{Aceptar} en el mensaje \hyperref[MSG12]{ \bf MSG12 Ingrese nombre}. \hyperref[W-PR-CU1.2:TA]{[Trayectoria alternativa A]}
	    
	    \item {[Sistema]} Verifica la información según la regla de negocio \hyperref[RN001]{RN-001} y \hyperref[RN002]{RN-002}.\hyperref[W-PR-CU1.2:TB]{[Trayectoria alternativa B]}
	    
	    \item {[Sistema]} Persiste el nuevo proyecto en el sistema.
	    
	    \item {[Sistema]} Muestra el mensaje \hyperref[MSG9]{\bf Operación exitosa} en la pantalla \textbf{\nameref{fig:MIUW-5}} y actualiza la lista de proyectos.

	\end{enumerate}
	Fin del caso de uso.
	
\paragraph{Trayectoria Alternativa A} \label{W-PR-CU1.2:TA}
	El actor cancela la operación.
	%para más trayectorias, copiar y pegar toda esta sección y en el comando siguiente cambiar por la letra correspondiente
	\begin{enumerate}[label=A\arabic*.]
		\item {[Actor]} Solicita cancelar la operación oprimiendo el botón \textit{Cancelar} en el mensaje \hyperref[MSG12]{\bf MSG12 Ingrese nombre}.
		\item {[Sistema]} Muestra la pantalla \textbf{\nameref{fig:MIUW-5}}.
	\end{enumerate}
	Fin de la trayectoria alternativa.

\paragraph{Trayectoria Alternativa B} \label{W-PR-CU1.2:TB}
	Datos no validos.
	%para más trayectorias, copiar y pegar toda esta sección y en el comando siguiente cambiar por la letra correspondiente
	\begin{enumerate}[label=B\arabic*.]
		\item {[Sistema]} Muestra el mensaje \hyperref[MSG1]{\bf MSG1 Datos no validos} en la pantalla \textbf{\nameref{fig:MIUW-5}}.
		\item {[Actor]} Continua con el paso 3 de la Trayectoria Principal.
	\end{enumerate}
	Fin de la trayectoria alternativa.
\newpage
\newpage
\section{Aplicación Web}
\subsection{Módulo usuarios}
\subsubsection{Diagrama de casos de uso}
En la figura \ref{fig:modulo-usuarios-web} se observa el diagrama de casos de uso correspondiente a este módulo.

\begin{figure}[H]
    \centering
    \includegraphics[width=400px]{capitulo4/web/modulo-usuarios-web.jpg}
    \caption{Diagrama de casos de uso del módulo Usuarios Web}
    \label{fig:modulo-usuarios-web}
\end{figure}

\input{capitulo4/web/usuarios/w-usr-cu1}
\newpage
\subsubsection{W-USR-CU1.1 Verificar cuenta}
Este caso de uso permite al usuario verificar su cuenta en el sistema.

\begin{longtable}{|J{5cm}|J{10.3cm}|}
	\hline
	\textbf{Nombre del Caso de Uso} &
		W-USR-CU1.1 Verificar cuenta \\ \hline
	\textbf{Objetivo} &
		Completar el registro de un usuario mediante la verificación de su cuenta \\ \hline
	\textbf{Actores} &
		Usuario no verificado \\ \hline 
	\textbf{Entradas} & 
		Ninguna \\ \hline 
	\textbf{Salidas} & 
		Ninguna \\ \hline
	\textbf{Precondiciones} &
		\begin{itemize}
		    \item El actor debe estar registrado en el sistema con un estatus no verificado
		\end{itemize} \\ \hline
	\textbf{Postcondiciones} &
	        La cuenta del actor será registrada con el estatus de verificada \\ \hline
	\textbf{Prioridad} & 
		Alta. \\ \hline
	\textbf{Reglas de Negocio} & 
		Ninguna \\ \hline

	% \caption{}
	%\label{desc:SUB-M-CU1}
\end{longtable}

\paragraph{Trayectoria Principal}
\label{W-USR-CU1.1}
	\begin{enumerate}
	    \item {[Sistema]} Verifica el código de autenticación.
	    
	    \item {[Sistema]} Cambia el estatus del actor como \textit{Verificado}.
	    
	    \item {[Sistema]} Muestra el mensaje \hyperref[MSG9]{MS9 Operación exitosa} en la pantalla \textbf{\nameref{fig:MIUW-1}}.
	\end{enumerate}
	Fin del caso de uso.
\newpage
\subsubsection{W-USR-CU2 Iniciar sesión}
Para poder hacer uso toda la funcionalidad de la aplicación es necesario que el usuario inicie sesión.
\begin{longtable}{|J{5cm}|J{10.3cm}|}
	\hline
	\textbf{Nombre del Caso de Uso} &
		W-USR-CU2 Iniciar sesión \\ \hline
	\textbf{Objetivo} &
		Permitir a un usuario validado y con cuenta validada el ingreso al sistema \\ \hline
	\textbf{Actores} &
		Usuario verificado \\ \hline 
	\textbf{Disparador} & 
		El actor requiere iniciar sesión \\ \hline 
	\textbf{Entradas} & 
		\begin{itemize}
				\item Correo
				\item Contraseña
		\end{itemize}\\ \hline 
	\textbf{Salidas} & 
		Ninguna. \\ \hline
	\textbf{Precondiciones} &
		\begin{itemize}
				\item El actor debe de tener una cuenta verificada
		\end{itemize} \\ \hline
	\textbf{Postcondiciones} &
		\begin{itemize}
			\item El actor pude hacer uso del resto de funcionalidades del sistema
		\end{itemize}\\ \hline
	\textbf{Prioridad} & 
		Media \\ \hline
	\textbf{Reglas de Negocio} & 
		\begin{itemize}
			\item \hyperref[RN001]{RN-001 Campos obligatorios}
			\item \hyperref[RN002]{RN-002 Datos correctos}
			\item \hyperref[RN006]{RN-006 Usuario verificado}
		\end{itemize} \\ \hline

	% \caption{}
	%\label{desc:SUB-M-CU1}
\end{longtable}
% Me faltan las referencias chidas
\paragraph{Trayectoria Principal}
	\begin{enumerate}
	    \item {[Actor]} Ingresa al sitio web.
	    \item {[Sistema]} Muestra la pantalla \hyperref[fig:MIUW-1]{\bf MIUW 1 Iniciar Sesión}.
	    \item {[Actor]} Ingresa la información solicitada en pantalla.
	    \item {[Actor]} Presiona el botón \textit{Iniciar sesión}.
	    \item {[Sistema]} Valida la información según la regla de negocio \hyperref[RN001]{RN-001} y \hyperref[RN002]{RN-002}. \hyperref[W-USR-CU2:TA]{[Trayectoria alternativa A]}
	    \item {[Sistema]} Autentica al usuario según la regla de negocio \hyperref[RN006]{RN-006}. \hyperref[W-USR-CU2:TB]{[Trayectoria alternativa B]}
	    \item {[Sistema]} Muestra la pantalla \hyperref[fig:MIUW-5]{\bf MIUW 5 Lista de Proyectos}.
	\end{enumerate}
	Fin del caso de uso.

\paragraph{Trayectoria Alternativa A} \label{W-USR-CU2:TA}
	La información ingresada por el actor no es valida.
	\begin{enumerate}[label=A\arabic*.]
		\item {[Sistema]} Muestra el mensaje \hyperref[MSG1]{\bf MSG1 Datos no validos} en la pantalla \hyperref[fig:MIUW-1]{\bf MIUW 1 Iniciar Sesión}.
	\end{enumerate}
	Fin de la trayectoria alternativa.
\paragraph{Trayectoria Alternativa B} \label{W-USR-CU2:TB}
	El usuario no ha verificado su cuenta.
	\begin{enumerate}[label=B\arabic*.]
		\item {[Sistema]} Muestra el mensaje \hyperref[MSG2]{\bf MSG2 Cuenta no verificada} en la pantalla \hyperref[fig:MIUW-1]{\bf{MIUW 1 Iniciar Sesión}}.
	\end{enumerate}
	Fin de la trayectoria alternativa.
\newpage
\input{capitulo4/web/usuarios/w-usr-cu3}
\newpage
\subsubsection{W-USR-CU4 Editar perfil}
La posibilidad de modificar los datos de una cuenta es indispensable para un usuario por lo que se proporciona un mecanismo para realizar esta tarea.
\begin{longtable}{|J{5cm}|J{10.3cm}|}
	\hline
	\textbf{Nombre del Caso de Uso} &
		W-USR-CU4 Editar perfil \\ \hline
	\textbf{Objetivo} &
		Permitir al actor editar los datos de su cuenta. \\ \hline
	\textbf{Actores} &
		Usuario verificado \\ \hline 
	\textbf{Disparador} & 
		El actor requiere modificar su información \\ \hline 
	\textbf{Entradas} & 
		\begin{itemize}
		    \item Nombre
		    \item Contraseña
		    \item Confirmación de contraseña
		    \item Grado de estudio
		\end{itemize} \\ \hline 
	\textbf{Salidas} & 
		\begin{itemize}
		    \item \hyperref[MSG9]{\bf MSG9 Operación exitosa}
		\end{itemize} \\ \hline
	\textbf{Precondiciones} &
		\begin{itemize}
				\item El actor debe de haber iniciado sesión
		\end{itemize} \\ \hline
	\textbf{Postcondiciones} &
		\begin{itemize}
			\item El actor modifica la información de su cuenta
		\end{itemize}\\ \hline
	\textbf{Prioridad} & 
		Baja \\ \hline
	\textbf{Reglas de Negocio} & 
		\begin{itemize}
		    \item \hyperref[RN001]{RN-001 Campos obligatorios}
		    \item \hyperref[RN002]{RN-002 Datos correctos}
		\end{itemize} \\ \hline

	% \caption{}
	%\label{desc:SUB-M-CU1}
\end{longtable}
% Me faltan las referencias chidas
\paragraph{Trayectoria Principal}
	\begin{enumerate}
	    \item {[Actor]} Presiona el la opción \textit{Cuenta} del menú \hyperref[fig:MIUW-7]{\bf MIUW 7 Menú de Opciones}.
	    \item {[Sistema]} Muestra la pantalla  \hyperref[fig:MIUW-3]{\bf MIUW 3 Editar Perfil}.
	    \item {[Actor]} Modifica la información.
	    \item {[Actor]} Presiona el botón guardar.
	    \item {[Sistema]} Valida la información según la regla de negocio \hyperref[RN001]{RN-001} y \hyperref[RN002]{RN-002}. \hyperref[W-USR-CU4:TA]{[Trayectoria Alternativa A]}
	    \item {[Sistema]} Persiste la información modificada.
	    \item {[Sistema]} Muestra el mensaje \hyperref[MSG9]{\bf MSG9 Operación exitosa} en la pantalla \hyperref[fig:MIUW-3]{\bf MIUW 3 Editar Perfil}.
	\end{enumerate}
	Fin del caso de uso.

\paragraph{Trayectoria Alternativa A} \label{W-USR-CU4:TA}
	Datos incorrectos.
	\begin{enumerate}[label=A\arabic*.]
		\item {[Sistema]} Muestra el mensaje \hyperref[MSG1]{\bf MSG11 Datos no validos} en la pantalla \hyperref[fig:MIUW-3]{\bf MIUW 3 Editar Perfil}.
	\end{enumerate}
	Fin de la trayectoria alternativa.
\newpage
\subsubsection{W-USR-CU5 Cerrar sesión}
Este caso de uso permite al usuario final cerrar su sesión en la aplicación.

\begin{longtable}{|J{5cm}|J{10.3cm}|}
	\hline
	\textbf{Nombre del Caso de Uso} &
		W-USR-CU5 Cerrar sesión \\ \hline
	\textbf{Objetivo} &
		Cerrar sesión en la aplicación \\ \hline
	\textbf{Actores} &
		Usuario verificado \\ \hline 
	\textbf{Disparador} & 
		El actor requiere cerrar su sesión \\ \hline 
	\textbf{Entradas} & 
		Ninguna. \\ \hline 
	\textbf{Salidas} & 
		Ninguna. \\ \hline
	\textbf{Precondiciones} &
		\begin{itemize}
		    \item El actor debe estar registrado en el sistema con un estatus de verificado
		    \item El actor debe haber iniciado sesión
		\end{itemize} \\ \hline
	\textbf{Postcondiciones} &
	        Se cierra la sesión \\ \hline
	\textbf{Prioridad} & 
		Media \\ \hline
	\textbf{Reglas de Negocio} & 
		Ninguna \\ \hline

	% \caption{}
	%\label{desc:SUB-M-CU1}
\end{longtable}

\paragraph{Trayectoria Principal}
\label{W-USR-CU5}
	\begin{enumerate}
	    \item {[Actor]} Solicita cerrar sesión oprimiendo el botón \textit{Cerrar Sesión} del menú \textbf{\nameref{fig:MIUW-7}}.
	    
	    \item {[Sistema]} Cierra la sesión del actor en el servidor.
	    
	    \item {[Sistema]} Muestra la pantalla \textbf{\nameref{fig:MIUW-1}}.
	\end{enumerate}
	Fin del caso de uso.

\newpage

\subsection{Módulo proyectos}
\subsubsection{Diagrama de casos de uso}
En la figura \ref{fig:modulo-proyectos-web} se observa el diagrama de casos de uso correspondiente a este módulo.

\begin{figure}[H]
    \centering
    \includegraphics[width=400px]{capitulo4/web/modulo-proyectos-web.jpg}
    \caption{Diagrama de casos de uso del módulo Proyectos Web}
    \label{fig:modulo-proyectos-web}
\end{figure}

\input{capitulo4/web/proyectos/w-pr-cu1}
\newpage
\input{capitulo4/web/proyectos/w-pr-cu1.1}
\newpage
\subsubsection{W-PR-CU1.1.1 Editar proyecto}
Este caso de uso permite al actor cambiar el nombre de algún proyecto previamente creado por él.

\begin{longtable}{|J{5cm}|J{10.3cm}|}
	\hline
	\textbf{Nombre del Caso de Uso} &
		W-PR-CU1.1.1 Editar proyecto \\ \hline
	\textbf{Objetivo} &
		Cambiar la información de algún proyecto \\ \hline
	\textbf{Actores} &
		Usuario verificado \\ \hline 
	\textbf{Disparador} & 
		El actor requiere cambiar la información de algún proyecto. \\ \hline 
	\textbf{Entradas} & 
		\begin{itemize}
				\item Nombre del proyecto
		\end{itemize}\\ \hline 
	\textbf{Salidas} & 
		\begin{itemize}
			\item \hyperref[MSG9]{\bf MSG9 Operación exitosa}
		\end{itemize} \\ \hline
	\textbf{Precondiciones} &
		\begin{itemize}
		    \item El actor debe estar registrado en el sistema con un estatus de validado
		    \item El usuario debe haber iniciado sesión
		    \item El proyecto debe de pertenecer al actor
		\end{itemize} \\ \hline
	\textbf{Postcondiciones} &
		Los datos del proyecto se modifican \\ \hline
	\textbf{Prioridad} & 
		Baja \\ \hline
	\textbf{Reglas de Negocio} & 
		\begin{itemize}
    		\item \hyperref[RN001]{RN-001 Campos obligatorios}
    		\item \hyperref[RN002]{RN-002 Datos correctos} 
    		\item \hyperref[RN005]{RN-005 Fecha de modificación}
    	\end{itemize}\\ \hline

	% \caption{}
	%\label{desc:SUB-M-CU1}
\end{longtable}

\paragraph{Trayectoria Principal}
\label{W-PR-CU1.1.1}
	\begin{enumerate}
	    \item {[Actor]} Presiona el botón \textit{Editar} en la pantalla \textbf{\nameref{fig:MIUW-6}}.
	    
	    \item {[Sistema]} Muestra un campo de texto con el actual nombre del proyecto en la pantalla \textbf{\nameref{fig:MIUW-6}}.
	    
	    \item {[Actor]} Ingresa el nuevo nombre del proyecto.
	    
	    \item {[Actor]} Presiona el botón aceptar \textit{Aceptar} de la pantalla \textbf{\nameref{fig:MIUW-6}}.  \hyperref[W-PR-CU1.1.1:TA]{[Trayectoria alternativa A]}
	    
	    \item {[Sistema]} Verifica los datos como se especifica en las reglas de negocio \hyperref[RN001]{RN-001} y \hyperref[RN002]{RN-002}. \hyperref[W-PR-CU1.1.1:TB]{[Trayectoria alternativa B]}
	    
	    \item {[Sistema]} Calcula la fecha de modificación del proyecto en el sistema según la regla de negocio \hyperref[RN005]{RN-005}.
	    
	    \item {[Sistema]} Persiste los cambios realizados.
	    
	    \item {[Sistema]} Muestra el mensaje \hyperref[MSG9]{\bf MSG9 Operación exitosa} en la pantalla \textbf{\nameref{fig:MIUW-6}}y actualiza la información mostrada. 

	\end{enumerate}
	Fin del caso de uso.
	
\paragraph{Trayectoria Alternativa A} \label{W-PR-CU1.1.1:TA}
	El actor cancela la operación.
	%para más trayectorias, copiar y pegar toda esta sección y en el comando siguiente cambiar por la letra correspondiente
	\begin{enumerate}[label=A\arabic*.]
		\item {[Actor]} Solicita cancelar la operación oprimiendo el botón \textit{Cancelar} en la pantalla \textbf{\nameref{fig:MIUW-6}}.
		\item {[Sistema]} Muestra la pantalla \textbf{\nameref{fig:MIUW-6}}.
	\end{enumerate}
	Fin de la trayectoria alternativa.

\paragraph{Trayectoria Alternativa B} \label{W-PR-CU1.1.1:TB}
	Datos no validos.
	%para más trayectorias, copiar y pegar toda esta sección y en el comando siguiente cambiar por la letra correspondiente
	\begin{enumerate}[label=B\arabic*.]
		\item {[Sistema]} Muestra el mensaje \hyperref[MSG1]{\bf MSG1 Datos no validos} en la pantalla \textbf{\nameref{fig:MIUW-6}}.
		\item {[Actor]} Continua con el paso 3 de la Trayectoria Principal.
	\end{enumerate}
	Fin de la trayectoria alternativa.
\newpage
\subsubsection{W-PR-CU1.1.2 Descargar proyecto}
El poder obtener un archivo de LaTeX para su uso es útil para el usuario por lo que se le brinda esta funcionalidad.

\begin{longtable}{|J{5cm}|J{10.3cm}|}
	\hline
	\textbf{Nombre del Caso de Uso} &
		W-PR-CU1.1.2 Descargar proyecto \\ \hline
	\textbf{Objetivo} &
		Generar un archivo de LaTeX para su descarga  \\ \hline
	\textbf{Actores} &
		Usuario verificado \\ \hline 
	\textbf{Disparador} & 
		El actor requiere descargar las traducciones de un proyecto \\ \hline 
	\textbf{Entradas} & 
		Ninguna \\ \hline 
	\textbf{Salidas} & 
		Archivo de LaTeX \\ \hline
	\textbf{Precondiciones} &
		\begin{itemize}
		    \item El actor debe haber iniciado sesión
		    \item Deben de existir traducciones para descargar el proyecto
		\end{itemize} \\ \hline
	\textbf{Postcondiciones} &
        \begin{itemize}
		    \item Se genera un archivo LaTeX
		\end{itemize} \\ \hline
	\textbf{Prioridad} & 
		Media \\ \hline
	\textbf{Reglas de Negocio} & 
		\begin{itemize}
		    \item \hyperref[RN007]{RN-007 Información necesaria para descargar un proyecto}
		\end{itemize} \\ \hline

	% \caption{}
	%\label{desc:SUB-M-CU1}
\end{longtable}

\paragraph{Trayectoria Principal}
	\begin{enumerate}
	    \item {[Actor]} Presiona el botón \textit{Descargar} de la pantalla  \textbf{\nameref{fig:MIUW-6}}.
	    \item {[Sistema]} Recupera la información necesaria para realizar esta operación según la regla de negocio \hyperref[RN007]{RN-007}. \hyperref[W-PR-CU1.1.2:TA]{[Trayectoria alternativa A]}
	    \item {[Sistema]} Genera un archivo LaTeX con las traducciones del proyecto seleccionado. \hyperref[W-PR-CU1.1.2:TB]{[Trayectoria alternativa B]}
	    \item {[Sistema]} Muestra el mensaje \hyperref[MSG9]{\bf MSG9 Operación exitosa} en la pantalla \textbf{\nameref{fig:MIUW-6}}.
	\end{enumerate}
	Fin del caso de uso.

\paragraph{Trayectoria Alternativa B} \label{W-PR-CU1.1.2:TA}
Falta información para realizar la operación
\begin{enumerate}[label=A\arabic*.]
    \item {[Sistema]} Muestra el mensaje \hyperref[MSG13]{\bf MSG13 Falta información}  en la pantalla \textbf{\nameref{fig:MIUW-6}}.
\end{enumerate}

\paragraph{Trayectoria Alternativa B} \label{W-PR-CU1.1.2:TB}
    No se pudo realizar la operación.
	\begin{enumerate}[label=B\arabic*.]
		\item {[Sistema]} Muestra el mensaje \hyperref[MSG10]{\bf MSG10 Operación fallida}  en la pantalla \textbf{\nameref{fig:MIUW-6}}.
	\end{enumerate}
	Fin de la trayectoria alternativa.
\newpage
\subsubsection{W-PR-CU1.1.3 Eliminar traducción}
Este caso de uso permite al usuario final eliminar una traducción previamente solicitada por él.

\begin{longtable}{|J{5cm}|J{10.3cm}|}
	\hline
	\textbf{Nombre del Caso de Uso} &
		W-PR-CU1.1.3 Eliminar traducción \\ \hline
	\textbf{Objetivo} &
		Eliminar una traducción del usuario \\ \hline
	\textbf{Actores} &
		Usuario verificado \\ \hline 
	\textbf{Disparador} & 
		El actor requiere eliminar una traducción \\ \hline 
	\textbf{Entradas} & 
		Ninguna \\ \hline 
	\textbf{Salidas} & 
		\begin{itemize}
			\item \hyperref[MSG9]{\bf MSG9 Operación exitosa}
		\end{itemize} \\ \hline
	\textbf{Precondiciones} &
		\begin{itemize}
		    \item El actor debe estar registrado en el sistema con un estatus de validado
		    \item El actor debe haber iniciado sesión
		    \item La traducción debe de existir en el sistema
		    \item La traducción debe de pertenecer al actor
		\end{itemize} \\ \hline
	\textbf{Postcondiciones} &
		\begin{itemize}
		    \item Al aceptar la eliminación de una traducción, toda su información correspondiente será eliminada del sistema
		\end{itemize} \\ \hline
	\textbf{Prioridad} & 
		Baja \\ \hline
	\textbf{Reglas de Negocio} & 
		Ninguna \\ \hline

	% \caption{}
	%\label{desc:SUB-M-CU1}
\end{longtable}

\paragraph{Trayectoria Principal}
\label{W-PR-CU1.1.3}
	\begin{enumerate}
	    \item {[Actor]} Presiona el botón \textit{Eliminar} de algún registro mostrado en la pantalla \textbf{\nameref{fig:MIUW-6}}.
	    
	    \item {[Sistema]} Muestra el mensaje \hyperref[MSG11]{\bf MSG11 Confirmación de operación}.
	    
	    \item {[Actor]} Acepta la eliminación oprimiendo el botón \textit{Aceptar}. \hyperref[W-PR-CU1.1.3:TA]{[Trayectoria alternativa A]}
	    
	    \item {[Sistema]} Verifica que la traducción pertenezca al usuario.
	    
	    \item {[Sistema]} Elimina toda la información correspondiente a la traducción.
	    
	    \item {[Sistema]} Muestra el mensaje \hyperref[MSG9]{\bf MSG9 Operación exitosa} en la pantalla \textbf{\nameref{fig:MIUW-6}} indicando que la operación se ha realizado exitosamente y actualiza la lista de traducciones.
	\end{enumerate}
	Fin del caso de uso.
	
\paragraph{Trayectoria Alternativa A} \label{W-PR-CU1.1.3:TA}
	El actor cancela la operación.
	%para más trayectorias, copiar y pegar toda esta sección y en el comando siguiente cambiar por la letra correspondiente
	\begin{enumerate}[label=A\arabic*.]
		\item {[Actor]} Solicita cancelar la operación oprimiendo el botón \textit{Cancelar}.
		\item {[Sistema]} Muestra la pantalla \textbf{\nameref{fig:MIUW-6}}.
	\end{enumerate}
	Fin de la trayectoria alternativa.

\newpage
\subsubsection{W-PR-CU1.1.4 Copiar traducción}
Una traducción que se realice por el sistema puede ser copiada sin necesidad de descargar todo el proyecto sobre el que se trabaja.
\begin{longtable}{|J{5cm}|J{10.3cm}|}
	\hline
	\textbf{Nombre del Caso de Uso} &
		W-PR-CU1.1.4 Copiar traducción \\ \hline
	\textbf{Objetivo} &
		Permitir copiar una traducción realizada  \\ \hline
	\textbf{Actores} &
		Usuario verificado \\ \hline 
	\textbf{Disparador} & 
		El actor requiere copiar una traducción \\ \hline 
	\textbf{Entradas} & 
		Ninguna \\ \hline 
	\textbf{Salidas} & 
		Copia de la traducción \\ \hline
	\textbf{Precondiciones} &
		\begin{itemize}
		    \item El actor debe haber iniciado sesión
		    \item Deben existir traducciones que copiar
		\end{itemize} \\ \hline
	\textbf{Postcondiciones} &
        \begin{itemize}
		    \item Se copia la traducción
		\end{itemize} \\ \hline
	\textbf{Prioridad} & 
		Media \\ \hline
	\textbf{Reglas de Negocio} & 
		Ninguna \\ \hline

	% \caption{}
	%\label{desc:SUB-M-CU1}
\end{longtable}

\paragraph{Trayectoria Principal}
	\begin{enumerate}
	    \item {[Actor]} Presiona el botón \textit{Copiar} de la pantalla  \textbf{\nameref{fig:MIUW-6}}.
	    \item {[Sistema]} Recupera la traducción.
	    \item {[Sistema]} Copia la traducción. \hyperref[W-PR-CU1.1.4:TA]{[Trayectoria alternativa A]}
	    \item {[Sistema]} Muestra el mensaje \hyperref[MSG9]{\bf MSG9 Operación exitosa} en la pantalla \textbf{\nameref{fig:MIUW-6}}.
	\end{enumerate}
	Fin del caso de uso.

\paragraph{Trayectoria Alternativa A} \label{W-PR-CU1.1.4:TA}
    No se pudo realizar la operación.
	\begin{enumerate}[label=A\arabic*.]
		\item {[Sistema]} Muestra el mensaje \hyperref[MSG10]{\bf MSG10 Operación fallida}  en la pantalla \textbf{\nameref{fig:MIUW-6}}.
	\end{enumerate}
	Fin de la trayectoria alternativa.
\newpage
\subsubsection{W-PR-CU1.1.5 Eliminar proyecto}
Este caso de uso permite al usuario final eliminar un proyecto previamente creado por él. 

\begin{longtable}{|J{5cm}|J{10.3cm}|}
	\hline
	\textbf{Nombre del Caso de Uso} &
		W-PR-CU1.1.5 Eliminar proyecto \\ \hline
	\textbf{Objetivo} &
		Eliminar un proyecto del usuario \\ \hline
	\textbf{Actores} &
		Usuario validado \\ \hline 
	\textbf{Disparador} & 
		El actor requiere eliminar un proyecto \\ \hline 
	\textbf{Entradas} & 
		Ninguna \\ \hline 
	\textbf{Salidas} & 
		\begin{itemize}
			\item \hyperref[MSG9]{\bf MSG9 Operación exitosa}
		\end{itemize} \\ \hline
	\textbf{Precondiciones} &
		\begin{itemize}
		    \item El actor debe estar registrado en el sistema con un estatus de verificado
		    \item El actor debe haber iniciado sesión
		    \item El proyecto debe de existir en el sistema
		    \item El proyecto debe de pertenecer al actor
		\end{itemize} \\ \hline
	\textbf{Postcondiciones} &
		\begin{itemize}
		    \item Al aceptar la eliminación del proyecto, todas sus traducciones asociadas serán eliminadas del sistema
		\end{itemize} \\ \hline
	\textbf{Prioridad} & 
		Media \\ \hline
	\textbf{Reglas de Negocio} & 
		Ninguna \\ \hline

	% \caption{}
	%\label{desc:SUB-M-CU1}
\end{longtable}

\paragraph{Trayectoria Principal}
\label{CUW_Eliminar_Proyecto}
	\begin{enumerate}
	    \item {[Actor]} Presiona el botón \textit{Eliminar Proyecto} de la pantalla \textbf{\nameref{fig:MIUW-6}}.
	    
	    \item {[Sistema]} Muestra el mensaje \hyperref[MSG11]{\bf MSG11 Confirmación de operación}.
	    
	    \item {[Actor]} Presiona el botón \textit{Aceptar}. \hyperref[W-PR-CU1.1.5:TA]{[Trayectoria alternativa A]}
	    
	    \item {[Sistema]} Verifica que el proyecto pertenezca al usuario.
	    
	    \item {[Sistema]} Elimina toda la información correspondiente al proyecto.
	    
	    \item {[Sistema]} Muestra el mensaje \hyperref[MSG9]{\bf MSG9 Operación exitosa} en la pantalla \textbf{\nameref{fig:MIUW-5}} y actualiza la lista de proyectos.
	\end{enumerate}
	Fin del caso de uso.
	
\paragraph{Trayectoria Alternativa A} \label{W-PR-CU1.1.5:TA}
	El actor cancela la operación.
	%para más trayectorias, copiar y pegar toda esta sección y en el comando siguiente cambiar por la letra correspondiente
	\begin{enumerate}[label=A\arabic*.]
		\item {[Actor]} Solicita cancelar la operación oprimiendo el botón \textit{Cancelar}.
		\item {[Sistema]} Muestra la pantalla \textbf{\nameref{fig:MIUW-6}}.
	\end{enumerate}
	Fin de la trayectoria alternativa.

\newpage
\subsubsection{W-PR-CU1.1.6.1 Calificar traducción}
Una traducción que se realice por el sistema se puede calificar para apoyar el algoritmo de traducción en un futuro, así como proporcionar retroalimentación al usuario de que tan util es el proyecto que se trabaja.
\begin{longtable}{|J{5cm}|J{10.3cm}|}
	\hline
	\textbf{Nombre del Caso de Uso} &
		W-PR-CU1.1.6.1 Calificar traducción \\ \hline
	\textbf{Objetivo} &
		Permitir la calificación de una traducción  \\ \hline
	\textbf{Actores} &
		Usuario verificado \\ \hline 
	\textbf{Disparador} & 
		El actor requiere calificar una traducción \\ \hline 
	\textbf{Entradas} & 
		\begin{itemize}
		    \item Calificación
		\end{itemize} \\ \hline
	\textbf{Salidas} &
	    \begin{itemize}
		    \item \hyperref[MSG9]{\bf MSG9 Operación exitosa}
		\end{itemize} \\ \hline
	\textbf{Precondiciones} &
		\begin{itemize}
		    \item El actor debe haber iniciado sesión
		    \item Deben existir traducciones que calificar
		\end{itemize} \\ \hline
	\textbf{Postcondiciones} &
        \begin{itemize}
		    \item Se califica la traducción
		    \item Se calcula la calificación del proyecto
		\end{itemize} \\ \hline
	\textbf{Prioridad} & 
		Baja \\ \hline
	\textbf{Reglas de Negocio} & 
		\begin{itemize}
		    \item \hyperref[RN004]{RN-004 Calificación proyecto}
		    \item \hyperref[RN005]{RN-005 Fecha de modificación}
		\end{itemize} \\ \hline

	% \caption{}
	%\label{desc:SUB-M-CU1}
\end{longtable}

\paragraph{Trayectoria Principal}
	\begin{enumerate}
	    \item {[Actor]} Presiona el número de estrellas como calificación
	    \item {[Sistema]} Muestra el mensaje \hyperref[MSG11]{\bf MSG11 Confirmación de operación} en la pantalla \textbf{\nameref{fig:MIUW-6}}.
	    \item {[Actor]} Presiona el botón \textit{Sí}. \hyperref[W-PR-CU1.1.6.1:TA]{[Trayectoria alternativa A]}
	    \item {[Sistema]} Calcula la calificación del proyecto según la regla de negocio \hyperref[RN004]{RN-004}.
	    \item {[Sistema]} Calcula la fecha de modificación según la regla de negocio \hyperref[RN005]{RN-005}.
	    \item {[Sistema]} Persiste la información  \hyperref[W-PR-CU1.1.6.1:TA]{[Trayectoria alternativa B]}
	    \item {[Sistema]} Muestra el mensaje \hyperref[MSG9]{\bf MSG9 Operación exitosa} en la pantalla \textbf{\nameref{fig:MIUW-6}}.
	\end{enumerate}
	Fin del caso de uso.

\paragraph{Trayectoria Alternativa A} \label{W-PR-CU1.1.6.1:TA}
    Se cancela la operación
	\begin{enumerate}[label=A\arabic*.]
	    \item {[Actor]} Presiona el botón \textit{No}.
		\item {[Sistema]} Muestra la pantalla \textbf{\nameref{fig:MIUW-6}}.
	\end{enumerate}
	Fin de la trayectoria alternativa.

\paragraph{Trayectoria Alternativa B} \label{W-PR-CU1.1.6.1:TB}
    No se pudo realizar la operación.
	\begin{enumerate}[label=A\arabic*.]
		\item {[Sistema]} Muestra el mensaje \hyperref[MSG10]{\bf MSG10 Operación fallida}  en la pantalla \textbf{\nameref{fig:MIUW-6}}.
	\end{enumerate}
	Fin de la trayectoria alternativa.
\newpage
\subsubsection{W-PR-CU1.2 Crear proyecto}
Este caso de uso permite al usuario final crear un nuevo proyecto para almacenar sus traducciones.

\begin{longtable}{|J{5cm}|J{10.3cm}|}
	\hline
	\textbf{Nombre del Caso de Uso} &
		W-PR-CU1.2 Crear proyecto \\ \hline
	\textbf{Objetivo} &
		Crear un nuevo proyecto en el cual se puedan almacenar traducciones \\ \hline
	\textbf{Actores} &
		Usuario verificado \\ \hline 
	\textbf{Disparador} & 
		El actor requiere crear un nuevo proyecto \\ \hline 
	\textbf{Entradas} & 
		\begin{itemize}
				\item Nombre del proyecto
		\end{itemize}\\ \hline 
	\textbf{Salidas} & 
		\begin{itemize}
			\item \hyperref[MSG9]{\bf MSG9 Operación exitosa}
		\end{itemize} \\ \hline
	\textbf{Precondiciones} &
		\begin{itemize}
		    \item El actor debe estar registrado en el sistema con un estatus de verificado.
		    \item El usuario debe haber iniciado sesión.
		\end{itemize} \\ \hline
	\textbf{Postcondiciones} &
		Ninguna. \\ \hline
	\textbf{Prioridad} & 
		Media. \\ \hline
	\textbf{Reglas de Negocio} & 
		\begin{itemize}
			\item \hyperref[RN001]{RN-001 Campos obligatorios}
			\item \hyperref[RN002]{RN-002 Datos correctos}
		\end{itemize} \\ \hline

	% \caption{}
	%\label{desc:SUB-M-CU1}
\end{longtable}

\paragraph{Trayectoria Principal}
\label{CUW_Crear_Proyecto}
	\begin{enumerate}
	    \item {[Actor]} Presiona el botón \textit{Crear Proyecto} en la pantalla \textbf{\nameref{fig:MIUW-5}}.
	    
	    \item {[Sistema]} Muestra el mensaje \hyperref[MSG12]{\bf MSG12 Ingrese nombre} en la pantalla \textbf{\nameref{fig:MIUW-5}} solicitando el nombre del nuevo proyecto.
	    
	    \item {[Actor]} Ingresa el nombre del nuevo proyecto.
	    
	    \item {[Actor]} Presiona el botón \textit{Aceptar} en el mensaje \hyperref[MSG12]{ \bf MSG12 Ingrese nombre}. \hyperref[W-PR-CU1.2:TA]{[Trayectoria alternativa A]}
	    
	    \item {[Sistema]} Verifica la información según la regla de negocio \hyperref[RN001]{RN-001} y \hyperref[RN002]{RN-002}.\hyperref[W-PR-CU1.2:TB]{[Trayectoria alternativa B]}
	    
	    \item {[Sistema]} Persiste el nuevo proyecto en el sistema.
	    
	    \item {[Sistema]} Muestra el mensaje \hyperref[MSG9]{\bf Operación exitosa} en la pantalla \textbf{\nameref{fig:MIUW-5}} y actualiza la lista de proyectos.

	\end{enumerate}
	Fin del caso de uso.
	
\paragraph{Trayectoria Alternativa A} \label{W-PR-CU1.2:TA}
	El actor cancela la operación.
	%para más trayectorias, copiar y pegar toda esta sección y en el comando siguiente cambiar por la letra correspondiente
	\begin{enumerate}[label=A\arabic*.]
		\item {[Actor]} Solicita cancelar la operación oprimiendo el botón \textit{Cancelar} en el mensaje \hyperref[MSG12]{\bf MSG12 Ingrese nombre}.
		\item {[Sistema]} Muestra la pantalla \textbf{\nameref{fig:MIUW-5}}.
	\end{enumerate}
	Fin de la trayectoria alternativa.

\paragraph{Trayectoria Alternativa B} \label{W-PR-CU1.2:TB}
	Datos no validos.
	%para más trayectorias, copiar y pegar toda esta sección y en el comando siguiente cambiar por la letra correspondiente
	\begin{enumerate}[label=B\arabic*.]
		\item {[Sistema]} Muestra el mensaje \hyperref[MSG1]{\bf MSG1 Datos no validos} en la pantalla \textbf{\nameref{fig:MIUW-5}}.
		\item {[Actor]} Continua con el paso 3 de la Trayectoria Principal.
	\end{enumerate}
	Fin de la trayectoria alternativa.
\newpage
\newpage
\section{Mensajes}
En esta sección se muestran los mensajes que se despliegan en pantalla para los diferentes casos de uso que se tienen.
\begin{enumerate}[label=MSG\arabic*.]
    \item \label{MSG1}
        \begin{description}
            \item \textbf{MSG1 Datos no validos}
            \item [Tipo] Error
            \item [Redacción] Los datos introducidos no son validos.
            %\item [Parametros]
             %\item [Ejemplo]
        \end{description}
    \item \label{MSG2}
        \begin{description}
            \item \textbf{MSG2 Cuenta no verificada}
            \item [Tipo] Error
            \item [Redacción] La cuenta aun no ha sido verificada.
        \end{description}
    \item \label{MSG3}
        \begin{description}
            \item \textbf{MSG3 Correo electrónico no registrado}
            \item [Tipo] Error
            \item [Redacción] No existe una cuenta con el correo electrónico introducido.
        \end{description}
    \item \label{MSG4}
        \begin{description}
            \item \textbf{MSG4 Correo electrónico ya registrado}
            \item [Tipo] Error
            \item [Redacción] El correo electrónico ya se ha utilizado.
        \end{description}
    \item \label{MSG5}
        \begin{description}
            \item \textbf{MSG5 Verifique su cuenta}
            \item [Tipo] Informativo
            \item [Redacción] Verifique su cuenta a través del correo electrónico que se le ha mandado.
        \end{description}
    \item \label{MSG6}
        \begin{description}
            \item \textbf{MSG6 Envio de correo de recuperación}
            \item [Tipo] Informativo
            \item [Redacción] Se ha enviado un correo electrónico para la recuperación de su contraseña.
        \end{description}
    \item \label{MSG7}
        \begin{description}
            \item \textbf{MSG7 No existen proyectos para mostrar}
            \item [Tipo] Informativo
            \item [Redacción] No hay proyectos que se puedan mostrar.
        \end{description}
    \item \label{MSG8}
        \begin{description}
            \item \textbf{MSG8 No se puede mostrar el proyecto}
            \item [Tipo] Error
            \item [Redacción] En este momento no se puede mostrar el proyecto.
        \end{description}
    \item \label{MSG9}
        \begin{description}
            \item \textbf{MSG9 Operación exitosa}
            \item [Tipo] Informativo
            \item [Redacción] La operación se ha realizado con éxito.
        \end{description}
    \item \label{MSG10}
        \begin{description}
            \item \textbf{MSG10 Operación fallida}
            \item [Tipo] Error
            \item [Redacción] La operación no se pudo realizar.
        \end{description}
    \item \label{MSG11}
        \begin{description}
            \item \textbf{MSG11 Confirmación de operación}
            \item [Tipo] Confirmación
            \item [Redacción] \hfill \\
            ¿Está seguro de $<$OPERACIÓN$>$ $<$ELEMENTO$>$ $<$NOMBRE\textunderscore ELEMENTO$>$?
            \item [Parametros] \hfill
                \begin{itemize} 
                    \item $<$OPERACIÓN$>$: Operación a realizar que se debe de confirmar.
                    \item $<$ELEMENTO$>$: Elemento sobre el cual se realizara la operación.
                    \item $<$NOMBRE\textunderscore ELEMENTO$>$: Nombre del elemento sobre el que se trabaja.
                \end{itemize}
            \item [Ejemplo] \hfill
                \begin{itemize}
                    \item ¿Está seguro de eliminar el proyecto tesis final?
                \end{itemize}
        \end{description}
    \item \label{MSG12}
        \begin{description}
            \item \textbf{MSG12 Ingrese el nombre}
            \item [Tipo] Confirmación
            \item [Redacción] Ingrese el nombre del proyecto.
        \end{description}
    \item \label{MSG13}
        \begin{description}
            \item \textbf{MSG13 Falta información}
            \item [Tipo] Error
            \item [Redacción] Falta información necesaria para realizar la operación.
        \end{description}
\end{enumerate}

\newpage
\section{Interfaces}
\subsection{Aplicación en Android}
\begin{figure}[h]
	\centering
	\includegraphics[width=200px]{capitulo4/imagenes/android/MIUA_1.png}
	\caption{MIUA 1 Iniciar Sesión}
	\label{fig:MIUA-1} %MOCK UP DE INTERFAZ DE USUARIO ANDROID
\end{figure}
\newpage
\begin{figure}[h]
	\centering
	\includegraphics[width=200px]{capitulo4/imagenes/android/MIUA_2.png}
	\caption{MIUA 2 Registro}
	\label{fig:MIUA-2} %MOCK UP DE INTERFAZ DE USUARIO ANDROID
\end{figure}
\newpage
\begin{figure}[h]
	\centering
	\includegraphics[width=200px]{capitulo4/imagenes/android/MIUA_3.png}
	\caption{MIUA 3 Recuperar Contraseña}
	\label{fig:MIUA-3} %MOCK UP DE INTERFAZ DE USUARIO ANDROID
\end{figure}
\newpage
\begin{figure}[h]
	\centering
	\includegraphics[width=200px]{capitulo4/imagenes/android/MIUA_4.png}
	\caption{MIUA 4 Menú de Opciones}
	\label{fig:MIUA-4} %MOCK UP DE INTERFAZ DE USUARIO ANDROID
\end{figure}
\newpage
\begin{figure}[h]
	\centering
	\includegraphics[width=200px]{capitulo4/imagenes/android/MIUA_5.png}
	\caption{MIUA 5 Lista de Proyectos}
	\label{fig:MIUA-5} %MOCK UP DE INTERFAZ DE USUARIO ANDROID
\end{figure}
\newpage
\begin{figure}[H]
	\centering
	\includegraphics[width=200px]{capitulo4/imagenes/android/MIUA_6.png}
	\caption{MIUA 6 Lista de Traducciones}
	\label{fig:MIUA-6} %MOCK UP DE INTERFAZ DE USUARIO ANDROID
\end{figure}
\newpage
\begin{figure}[H]
	\centering
	\includegraphics[width=200px]{capitulo4/imagenes/android/MIUA_7.png}
	\caption{MIUA 7 Editar Perfil}
	\label{fig:MIUA-7} %MOCK UP DE INTERFAZ DE USUARIO ANDROID
\end{figure}

\newpage
\subsection{Aplicación en Web}
\begin{figure}[H]
	\centering
	\includegraphics[width=300px]{capitulo4/imagenes/web/IUW_1.png}
	\caption{MIUW 1 Iniciar Sesión}
	\label{fig:MIUW-1}
\end{figure}

\begin{figure}[H]
	\centering
	\includegraphics[width=300px]{capitulo4/imagenes/web/IUW_2.png}
	\caption{MIUW 2 Registro}
	\label{fig:MIUW-2}
\end{figure}

\begin{figure}[H]
	\centering
	\includegraphics[width=300px]{capitulo4/imagenes/web/IUW_3.png}
	\caption{MIUW 3 Editar Perfil}
	\label{fig:MIUW-3}
\end{figure}

\begin{figure}[H]
	\centering
	\includegraphics[width=300px]{capitulo4/imagenes/web/IUW_4.png}
	\caption{MIUW 4 Recuperar Contraseña}
	\label{fig:MIUW-4}
\end{figure}

\begin{figure}[H]
	\centering
	\includegraphics[width=300px]{capitulo4/imagenes/web/IUW_5.png}
	\caption{MIUW 5 Lista de Proyectos}
	\label{fig:MIUW-5}
\end{figure}

\begin{figure}[H]
	\centering
	\includegraphics[width=300px]{capitulo4/imagenes/web/IUW_6.png}
	\caption{MIUW 6 Lista de Traducciones}
	\label{fig:MIUW-6}
\end{figure}

\begin{figure}[H]
	\centering
	\includegraphics[width=300px]{capitulo4/imagenes/web/IUW_7.png}
	\caption{MIUW 7 Menú de Opciones}
	\label{fig:MIUW-7}
\end{figure}
