\newglossaryentry{imgCap}
{
    name=Image Captioning,
    description={Área de investigación que combina el análisis de imágenes con el procesamiento del lenguaje natural.},
    category={common}
}

\newglossaryentry{grafo}
{
    name=Grafo,
    description={Es un conjunto de objetos llamados vértices unidos por enlaces llamados aristas, que permiten representar relaciones binarias entre sus elementos.},
    category={common}
}

\newglossaryentry{Latex}
{
    name=LaTeX,
    description={Es un sistema de composición de textos, orientado a la creación de documentos escritos que presenten una alta calidad tipográfica.},
    category={common}
}

\newglossaryentry{JDK}
{
    name=JDK,
    description={De sus siglas en inglés \textit{"Java Development Tool Kit"} es la utilería de desarrollo del lenguaje Java.},
    category={common}
}

\newglossaryentry{SDK}
{
    name=SDK,
    description={Kit de Desarrollo de Software, es un conjunto de herramientas que ofrece el fabricante de algún sistema o lenguaje.},
    category={common}
}

\newglossaryentry{API}
{
    name=API,
    description={Una API es un conjunto de definiciones y protocolos que se utiliza para desarrollar e integrar el software de las aplicaciones. API significa interfaz de programación de aplicaciones.},
    category={common}
}

\newglossaryentry{REST}
{
    name=REST,
    description={Es una interfaz para conectar varios sistemas basados en el protocolo HTTP (uno de los protocolos más antiguos) y nos sirve para obtener y generar datos y operaciones, devolviendo esos datos en formatos muy específicos, como XML y JSON.},
    category={common}
}

\newglossaryentry{Algoritmo}
{
    name=Algoritmo,
    description={Conjunto de instrucciones finitas y no ambigüas que resuelven un problema.},
    category={common}
}

\newglossaryentry{MVC}
{
    name=MVC,
    description={Modelo Vista Controlador, es un patrón de diseño que separa a una aplicación en tres partes: la presentación, la lógica y los datos.},
    category={common}
}

\newglossaryentry{Servidor}
{
    name=Servidor,
    description={Una computadora destinada a escuchar y atender peticiones de otras computadoras referidas como clientes.},
    category={common}
}

\newglossaryentry{POJO}
{
    name=POJO,
    description={Una instancia de una clase que no implementa o extiende nada en especial. Para los programadores Java sirve para enfatizar el uso de clases simples y que no dependen de un framework en especial.},
    category={common}
}

\newglossaryentry{GPU}
{
    name=GPU,
    description={Una unidad de procesamiento gráfico es un coprocesador dedicado al procesamiento de gráficos u operaciones de coma flotante, para aligerar la carga de trabajo del procesador central en aplicaciones como los videojuegos o machine learning.},
    category={common}
}

\newglossaryentry{ORM}
{
    name=ORM,
    description={De sus siglas en inglés \textit{"Object Relational Model"}, es un modelo de programación que permite mapear las estructuras de una base de datos relacional sobre una estructura lógica de entidades con el objeto de simplificar y acelerar el desarrollo de las aplicaciones.},
    category={common}
}

\newglossaryentry{HTTP}
{
    name=HTTP,
    description={De sus siglas en inglés \textit{"Hypertext Transfer Protocol"}, es el nombre de un protocolo el cual nos permite realizar una petición de datos y recursos, como pueden ser documentos HTML. Es la base de cualquier intercambio de datos en la Web.},
    category={common}
}

\newglossaryentry{Framework}
{
    name=Framework,
    description={Es un conjunto de prácticas, conceptos, módulos y librerías que facilitan el desarrollo de una aplicación.},
    category={common}
}
