\chapter{Conclusiones}
La detección de expresiones matemáticas escritas a mano es un problema difícil debido a la estructura jerárquica y bidimensional que presentan y a la ambigüedad inherente de la escritura a mano haciendo de este un problema de investigación abierto. Conscientes de esto, vimos la oportunidad de realizar un prototipo que busque dar solución al problema y a su vez tenga una utilidad para los usuarios de \LaTeX.

Los objetivos específicos planteados para la aplicación web y android se lograron alcanzar a través de un correcto análisis y diseño lo cual permitió que su desarrollo fuera lo más ágil posible al elegir las mejores tecnologías para su elaboración de acuerdo a las necesidades que se tenían.

Por otro lado, utilizar el algoritmo Sauvola para la binarización por sobre Otsu fue buena elección ya que presentó mejores resultados en las diferentes pruebas que se realizaron, permitió hacer que la imágenes que se toman con la cámara del teléfono fueran semejantes a las que se presentan en el conjunto de entrenamiento por lo que este objetivo especifico se alcanzó.

Con respecto al módulo de traducción de expresiones matemáticas, este objetivo no se concretó ya que la precisión requerida no es suficiente para llevarlo a un ambiente de producción debido a que dicha precisión está directamente afectada por el conjunto de entrenamiento como se menciona a lo largo de este trabajo. No consideramos que los resultados sean malos, debido a que se alcanzó una precisión como la que presentan los artículos que sirvieron como base para la elaboración de este trabajo. En el caso de las expresiones escritas a mano un conjunto con una mayor cantidad de imágenes podría mejorar significativamente los resultados del trabajo realizado, esto de acuerdo a la precisión obtenida con imágenes renderizadas por computadora en cuyo caso el conjunto de entrenamiento es al menos 10 veces mayor a su contraparte de expresiones escritas a mano, llegando a la conclusión de que existe una correlación directa entre la precisión y el tamaño del conjunto de entrenamiento, además, a pesar de proponer otras variantes de la red neuronal éstas no lograron mejorar nuestros resultados para expresiones escritas a mano.

Finalmente y tomando en cuenta lo anterior, varios objetivos plateados en un inicio del trabajo terminal se lograron cubrir, sin embargo, el reconocimiento y traducción de las expresiones matemáticas escritas a mano es un problema aún no resuelto, como varios artículos citados a lo largo de este trabajo lo demuestran y por ende se puede trabajar más en buscar nuevas alternativas que brinden mejores resultados a los obtenidos en el presente trabajo terminal.

\chapter{Trabajo futuro}

La tarea primordial consiste en incrementar el conjunto de entrenamiento de expresiones matemáticas escritas a mano para que la cantidad de ejemplos disponibles sea similar al de conjunto de entrenamiento Harvard 100k.

Una forma de alcanzar este nuevo objetivo es el proporcionar al usuario la oportunidad de brindar retroalimentación a través de correcciones que se hagan a traducciones hechas por el sistema.

Esto se puede implementar en la parte android y el la parte web para que la funcionalidad de estas dos aplicaciones aumente. Además de buscar el llevar a producción estas dos aplicaciones una vez que se tenga un mejor resultado.

El proporcionar la oportunidad para un ambiente colaborativo en la gestión de proyectos para muchos usuarios y no solo la gestión de proyectos individuales agregará mayor valor al trabajo.

Probar el modelo desarrollado en el concurso CROHME de donde se obtuvo el conjunto de entrenamiento, el objetivo de dicho concurso se centra en el reconocimiento de expresiones matemáticas escritas a mano.

El reconocer expresiones matemáticas renderizadas por computadora es algo que se puede implementar ya que esto incrementaría la funcionalidad de la aplicación al volverla más versátil.
