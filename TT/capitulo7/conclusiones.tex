\chapter{Conclusiones}
La detección de expresiones matemáticas es un problema

El utilizar el algoritmo Sauvola para la binarización por sobre Otsu fue buena elección ya que presento mejores resultados en las diferentes pruebas que se realizaron, permitió hacer que la imágenes que se toman con la cámara del teléfono fueran semejantes a las que se presentan en el conjunto de entrenamiento.\\ll

Con respecto al modulo de traducción de expresiones matemáticas, la precisión de dicho modulo está directamente afectada por el conjunto de entrenamiento. En el caso de las expresiones escritas a mano un conjunto con una mayor cantidad de imágenes podría mejorar significativamente los resultados del trabajo realizado esto de acuerdo a la precisión obtenida con imágenes renderizadas por computadora en cuyo caso el conjunto de entrenamiento es al menos 10 veces mayor a su contraparte de expresiones escritas a mano.\\

Finalmente y tomando en cuenta lo anterior, es evidente que varios puntos plateados en un inicio del trabajo terminal se lograron cubrir, sin embargo, la traducción de las expresiones matemáticas es algo que se tiene que mejorar.

\chapter{Trabajo futuro}

%Con los resultados obtenidos se puede plantear incrementar el conjunto de entrenamiento de expresiones matemáticas escritas a mano para que la cantidad de ejemplos disponibles sea similar al de conjunto de entrenamiento Harvard 100k, esto debido a que es el principal problema que se detectó a lo largo de la realización del trabajo terminal y que pese a las variantes que se probaron no se logro solventar este problema.\\

Al realizar esto se podrían obtener mejores resultados, ya que como se demostró con este segundo conjunto de entrenamiento el hecho de contar con una mayor cantidad de elementos nos proporciona un mejor resultado.\\

Una forma de alcanzar este nuevo objetivo podría ser el proporcionar al usuario la oportunidad de brindar retroalimentación al sistema, esto es algo que se planteo como una posibilidad desde TT1 y que ahora pude ser una oportunidad para mejorar el funcionamiento del sistema.