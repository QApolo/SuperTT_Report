\chapter{Pruebas del sistema}

En este capítulo se detallan las diversas pruebas realizadas a los módulos desarrollados con el fin de verificar y validar su correcto funcionamiento.

\section{Pruebas unitarias}
\subsection{Android}
Android es un sistema operativo para móviles que en la actualidad es desarrollado por Google. Esta principalmente pensado para dispositivos con pantalla táctil como smartphones y tablets. Y que en la actualidad junto con iOS son las principales opciones en cuanto a sistemas operativos para teléfonos móviles. Es por esto que en la Tabla \ref{tbl:comparativa-moviles} se hace una comparativa de las principales características que se tomaron en cuenta para la elección de sistema operativo de dispositivos móviles que se usaría en este trabajo.
\begin{center}
    \begin{longtable}{|J{3cm}|J{5cm}|J{5cm}|}
    \hline
    \textbf{Características} & \textbf{Android} & \textbf{iOS} \\ \hline
    Núcleo & UNIX & UNIX \\ \hline
    Lenguajes de desarrollo & Java, C, C++, Kotlin & Swift, C, C++, Objective-C \\ \hline
    Ambiente de desarrollo & Android Studio, disponible en Windows, Linux y MacOS & Xcode, solo disponible en MacOS \\ \hline
    Complejidad de desarrollo & Medio complejo debido a la gran cantidad de diferencias entre dispositivos y versiones de sistema operativo & Poco compleja debido a que hay poca diversidad de versiones sistemas operativos y dispositivos \\ \hline
    Tiempo de desarrollo & Suele tomar 30\%-40\% más que para iOS \cite{ddi} & Depende de la complejidad del desarrollo \\ \hline
    Tiempo de despliegue & Es rápido desplegar una aplicación de Android debido a los test automáticos que se realizan sobre esta para su verificación & Es lento su despliegue debido a que la verificación es manual \\ \hline
    Cantidad del mercado & La cantidad de usuarios es mayor, tan solo el primer cuarto del 2017 el 86.1\% de los teléfonos vendidos fueron Android \cite{ddi} & Su mercado es menor al de Android \\ \hline
    Código abierto & El kernel, UI y algunas aplicaciones estándar son de código abierto & El kernel no es de código abierto pero esta basado en Darwin OS que es de código abierto \\ \hline
    Ultima versión & Android 10 (Septiembre 3, 2019) & iOS 13 (Septiembre 19, 2019) \\ \hline
    Seguridad & Actualizaciones de seguridad bastante regulares. Sin embargo, debido a los fabricantes dichas actualizaciones pueden demorarse en ser aplicadas. También se pueden instalar aplicaciones externas a la Play Store por lo que esto puede generar problemas de seguridad & Pocas actualizaciones, las amenazas de seguridad son pocas debido a que descargar aplicaciones fuera de la App Store es complicado. \\ \hline
    \caption{Tabla comparativa de sistemas operativos de dispositivos móviles}
    \label{tbl:comparativa-moviles}
    \end{longtable}
\end{center}

Android nos brinda un ambiente de desarrollo más flexible que el que presenta iOS debido a las características listadas con anterioridad. Es cierto que el desarrollo puede ser más tardado sin embargo para este proyecto el optar por iOS nos generaría problemas debido a que no solo la plataforma de desarrollo es más cerrada sino que el tiempo de desarrollo seria mayor debido a la curva de aprendizaje de Swift y Objective-C a la cual nos enfrentaríamos.

Finalmente, la cantidad del mercado que tiene Android es significativamente mayor que la que tiene iOS por ende este es un gran punto a considerar. Es por todas estas cuestiones que se opto por desarrollar el trabajo para la plataforma de Android.
\subsection{Aplicación web}

Para realizar pruebas unitarias a la aplicación web desarrollado con Django se utilizo la herramienta que permite analizar que partes del código de un programa se están ejecutando y con ello determinar que bloques de código se deben de someter a pruebas.

Al utilizar coverage.py sobre el proyecto de Django se obtuvieron los resultados que se muestran en la figura \ref{fig:coverage}. El reporte que coverage.py arroja muestra la cantidad de código que se tiene que probar, el código que falta por probar, el excluido y el que se tienen cubierto con las pruebas.

Se logro cubrir el 100\% del código y se excluyeron algunas partes debido a que forman parte de los archivos de configuración de Django y porque fueron pruebas complicadas de elaborar.

\begin{figure}[H]
	\centering
	\includegraphics[width=250px]{capitulo6/unitarias/img/coverage.png}
	\caption{Reporte del código sometido a pruebas}
	\label{fig:coverage}
\end{figure}

Para realizar pruebas sobre estos bloques de código encontrados, se utilizo el módulo de pruebas unitarias con el que cuenta Django y se realizaron pruebas sobre los modelos, vistas y las clases de utilitaria que se desarrollaron.

Como se puede apreciar en la figura \ref{fig:pruebasDjango}, se realizaron 777 pruebas sobre modelos, vistas y utilitaria, al final siendo exitosas cada una de ellas.

\begin{figure}[H]
	\centering
	\includegraphics[width=450px]{capitulo6/unitarias/img/pruebasDjango.png}
	\caption{Resultado de las pruebas de Django}
	\label{fig:pruebasDjango}
\end{figure}

La forma en la que se elaboraron estas pruebas fue la siguiente.

\subsubsection{Pruebas sobre los modelos}

Un ejemplo de las pruebas sobre modelos es el siguiente código, en el cual se prueba el modelo del usuario y en cada uno de los métodos que comienzan con la palabra test determinan los diferentes casos de prueba a elaborar, que en este ejemplo son sobre el guardado de información de un usuario.

\lstinputlisting[language=Python]{capitulo6/unitarias/codigo/test_models.py}

En las primeras dos pruebas se utiliza el método assertEquals el cual permite comparar el resultado de una prueba con el esperado para determinar si la prueba fue exitosa.

Para las siguientes tres pruebas como se espera que se arroje una excepción se utiliza la declaración with para determinar si la excepción esperada ocurrió y con ello asegurar que la prueba fue exitosa, si esto no ocurre la prueba no es correcta. 

\subsubsection{Pruebas sobre los vistas}

\lstinputlisting[language=Python]{capitulo6/unitarias/codigo/test_views.py}

\subsubsection{Pruebas sobre las clases de utilitaria}

\lstinputlisting[language=Python]{capitulo6/unitarias/codigo/test_util.py}

\subsection{Análisis de las pruebas}
El desarrollar las pruebas unitarias fue necesario ya que se valida el funcionamiento individual de cada parte del sistema, como se presentó con anterioridad todas las pruebas fueron exitosas por lo que cada parte del código cumple su cometido. Es importante señalar que de cambiar el código no debería de cambiar el resultado de las pruebas debido a que la funcionalidad se mantiene igual por lo que en caso de que las pruebas ahora dieran resultados erróneos esto nos indicaría que los cambios no fueron los correctos esto es otro beneficio que se tienen al realizar este tipo de pruebas.
%\begin{enumerate}
%    \item Unitarias
%    \item De integración
%    \item De los requerimientos no funcionales
%    \item Y principalmente con la arquitectura de la red neuronal y el conjunto de entrenamiento y con inputs esperados por el usuario
%\end{enumerate}{}